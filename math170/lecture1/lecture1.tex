\documentclass[hidelinks]{article}
\usepackage[a4paper, total={7in, 10in}, top=0.25in]{geometry}
\usepackage[dvipsnames]{xcolor}
\usepackage{amsmath}
\usepackage{tikz}
\usepackage{tkz-euclide}
\usepackage[unicode]{hyperref}
\usepackage[all]{hypcap}
\usepackage{fancyhdr}
\usepackage{amsfonts}
\usepackage[utf8]{inputenc}
\DeclareUnicodeCharacter{2212}{-}
\usepackage{amsmath}
\usepackage{array}
\mathchardef\mhyphen="2D % Define a "math hyphen"
\newcommand\rnumber{\mathop{r\mhyphen number}}

\usetikzlibrary{angles,calc, decorations.pathreplacing}

\definecolor{carminered}{rgb}{1.0, 0.0, 0.22}
\definecolor{capri}{rgb}{0.0, 0.75, 1.0}
\definecolor{brightlavender}{rgb}{0.75, 0.58, 0.89}
\title{\textbf{Math 170E Lecture 1}}
\author{Allan Zhang}
\date{March 31, 2025}
\begin{document}
\hypersetup{bookmarksnumbered=true,}
\definecolor{darkspringgreen}{rgb}{0.09, 0.45, 0.27}
\definecolor{darkseagreen}{rgb}{0.56, 0.74, 0.56}
\definecolor{green(munsell)}{rgb}{0.0, 0.66, 0.47}
\pagecolor{white}
\color{black}
\maketitle
\section*{Basic Notions of Probability Theory}
Consider an experiment with unknown probabilistic outcome, set of all possible outcomes $S$ is called the \textit{sample space}

\begin{itemize}

    \item[] Ex) For the experiment where a coin is flipped three times, 
\[
    S = \{\text{TTT, TTH, THH, THT, HTT, HTH, HHT, HHH}\}
\]

\item[] Elements $s\in S$ are outcomes, while $A \subset S$ are events
\end{itemize}
We consider a function $\mathbb{P}:\{A \subset S\} \rightarrow [0, 1]$, which heuristically assigns to each event $A$ the probability $\mathbb{P}(A)$ that it occurs. P also satisfies some natural properties like $\mathbb{P}(A\sqcup B)$
\[
A \sqcup B = A \cup B \text{ in the case that } A \cap B = \emptyset
\]

\subsection*{Conditional Probability}
\[\mathbb{P} := \frac{\mathbb{P} = (A \cap B)}{\mathbb{P}(B)}\]
This equation represents the conditional probability of event $A$ occurring given that event $B$ has already occurred. In other words, it's the probability that $A$ happens under the condition that $B$ is true.


\subsection*{Example Problem}
Given that you flip a coin 3 times, what are the chances that 

1. You get exactly 2 heads?

2. You get exactly 2 heads if the first flip is heads 

3. The first flip was heads if you know there were exactly two heads 
\vspace{0.3cm}

Problem 1
\[
    \text{The total number of combinations is 2 * 2 * 2 = 8}
\]
\[
    \text{Writing out the sample space, we know that the possible outcomes are:}
\]
\[S = \{\text{TTT, TTH, THH, THT, HTT, HTH, HHT, HHH}\}\]
\[
    \text{There are 3 cases where there are 2 heads, so the answer is } \boxed{\frac{3}{8}}
\]
\vspace{0.3cm}

Problem 2 
\begin{center}
    If we look at the cases where the first flip was heads, we are left with $\{\text{HTT, HHT, HTH, HHH}\}$. Then out of these 4 choices, we only have 2 cases where there are 2 heads, so the solution is $\boxed{\frac{1}{2}}$
\end{center}

Problem 2 
\begin{center}
    The subset which includes all elements with 2 heads is $\{THH, HTH, HHT\}$. From this, we can observe that there are two elements where the combination begins with heads, so the solution is $\boxed{\frac{2}{3}}$
\end{center}
\end{document}

\noindent
