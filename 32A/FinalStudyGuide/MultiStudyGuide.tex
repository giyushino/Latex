\documentclass[hidelinks]{article}
\usepackage[a4paper, total={7in, 10in}]{geometry}
\usepackage[dvipsnames]{xcolor}
\usepackage{amsmath}
\usepackage{tikz}
\usepackage{tkz-euclide}
\usepackage[unicode]{hyperref}
\usepackage[all]{hypcap}
\usepackage{fancyhdr}
\usepackage{amsfonts}

\usetikzlibrary{angles,calc, decorations.pathreplacing}

\definecolor{carminered}{rgb}{1.0, 0.0, 0.22}
\definecolor{capri}{rgb}{0.0, 0.75, 1.0}
\definecolor{brightlavender}{rgb}{0.75, 0.58, 0.89}
\title{\textbf{MATH 32A Final Study Guide}}
\author{Allan Zhang}
\date{The date doesn't matter}
\begin{document}
\hypersetup{bookmarksnumbered=true,}
\definecolor{darkspringgreen}{rgb}{0.09, 0.45, 0.27}
\definecolor{darkseagreen}{rgb}{0.56, 0.74, 0.56}
\definecolor{green(munsell)}{rgb}{0.0, 0.66, 0.47}
\pagecolor{white}
\color{black}
\maketitle

\section{Quick Reminder on Single Variable Calculus}
\begin{itemize}
	\item Applied to functions of one variable
	\item Classic example is business owner, attempts to maximize profit $\rightarrow$ function of single variable x, 
	\begin{itemize}
		\item Profit = $f(x)$ where x = set price of object 
	\end{itemize}
	\item More realistic model is $f(x_1, x_2, x_3)$ where $x_1 = $ price, $x_2 = $ money spent on advertisment, $x_3 =$ market flux 
\end{itemize}

\section{Vectors}
\begin{itemize}
	\item Def) Plane: $\mathbb{R}^2$: points are $(a, b)$ where $a \text{ and } b $ are real numbers
	\item Def) Vector: $P, Q$ points $\rightarrow$ $\vec{PQ}$ vector starts at $P \text{, ends at } Q$ 
\end{itemize}
\begin{tikzpicture}
	\draw[thin,gray!40] (-4,-4) grid (4,4);
	\fill (3, 2) circle (1pt) node[right] {$(a, b)$};
	\draw[->] (0,0) -- (3,2);
	\draw[<->] (-4,0)--(4,0) node[right]{$x$};
  	\draw[<->] (0,-4)--(0,4) node[above]{$y$};
	\draw (3, 0.1) -- (3, -0.1) node[below] {$a$};
	\draw(0.1, 2) -- (-0.1, 2) node[left] {$b$};
\end{tikzpicture}
\begin{itemize}
	\item Take vector $\vec{PQ}$ where $P = (a, b) \quad Q = (c, d)$ 
	\item[] \quad Components of $\vec{PQ}$ are $\langle c - a, d -b \rangle$ = $\begin{bmatrix} c-a \\ d- b \end{bmatrix}$
\end{itemize}
\begin{tikzpicture}
	\draw[thin,gray!40] (-4,-4) grid (4,4);
	\fill (3, 2) circle (0pt) node[right] {$\vec{v}$};
	\draw[->] (0,0) -- (3,2); node[above];
	\draw[<->] (-4,0)--(4,0) node[right]{$x$};
  	\draw[<->] (0,-4)--(0,4) node[above]{$y$}
	\draw[->] (0,-3) -- (3, -1) 
	\fill (2.8, -1) circle (0pt) node[left] {Translation of $\vec{v}$};
\end{tikzpicture}

Translating a vector does not change the components, only the basepoints

The component only takes into consideration the magnitude and direction 
\subsection{Properities of Vectors in 2-D} 
\begin{itemize}
	\item Length of vector $\langle a, b \rangle = \sqrt{a^2+b^2}$
	\item $\vec{v} = \langle a, b \rangle \quad \vec{w} = \langle c, d \rangle$
\end{itemize}
\end{document}


