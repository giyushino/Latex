\documentclass[hidelinks]{article}
\usepackage[a4paper, total={7in, 10in}]{geometry}
\usepackage[dvipsnames]{xcolor}
\usepackage{amsmath}
\usepackage{tikz}
\usepackage{tkz-euclide}
\usepackage[unicode]{hyperref}
\usepackage[all]{hypcap}
\usepackage{fancyhdr}
\usepackage{amsfonts}

\usetikzlibrary{angles,calc, decorations.pathreplacing}

\definecolor{carminered}{rgb}{1.0, 0.0, 0.22}
\definecolor{capri}{rgb}{0.0, 0.75, 1.0}
\definecolor{brightlavender}{rgb}{0.75, 0.58, 0.89}
\title{\textbf{MATH 32A Problem Set 8}}
\author{Allan Zhang}
\date{November 20th, 2024}
\begin{document}
\hypersetup{bookmarksnumbered=true,}
\definecolor{darkspringgreen}{rgb}{0.09, 0.45, 0.27}
\definecolor{darkseagreen}{rgb}{0.56, 0.74, 0.56}
\definecolor{green(munsell)}{rgb}{0.0, 0.66, 0.47}
\pagecolor{white}
\color{black}
\maketitle


\section{Question 1}
Suppose $h(x, y) = 7y \sin(x) + 19y$ is a function of two variables $x$ and $y$.
Suppose that $g(s, t) = 4t^2 + s$, and $f(s, t) = \frac{e^t}{s}$ are two functions of $s$ and $t$. Define a
composite function $P(s, t) = h(g(s, t), f(s, t))$. What are $\frac{\partial s}{\partial t}$ and $\frac{\partial P}{\partial t}$ ? 

\[
	P(s, t) = h(4t^2+s, \frac{e^t}{s}) = 7(\frac{e^t}{s})sin(4t^2+s) + 19(\frac{e^t}{s})
\]

\[
	\text{In this context, s is not a function of t so } \frac{\partial s}{\partial t} = 0 
\]

\[
	\frac{\partial P}{\partial t} = \frac{\partial h}{\partial x}\frac{\partial g}{\partial t} + \frac{\partial h}{\partial y} \frac{\partial f}{\partial t} = 7y\cos(x) \cdot 8t + (7\sin{x} + 19) \frac{e^t}{s}
\]

\[
	\text{Remember that } x = g(s, t) \text{ and } y = f(s, t)
\]
\[
	\frac{\partial P}{\partial t} = 7(\frac{e^t}{s})\cos(4t^2 + s) + (7\sin(4t^2+s) + 19)\frac{e^t}{s} 
\]
\newpage

\section{Question 2}
Suppose there exists a magic cylinder whose dimensions change based on the
temperature $t$ (in degrees Fahrenheit) and the and the absolute humidity $h$ (in grams per
cubic meter): the base of the cylinder has radius $t + \frac{\sin(h)}{t}$, and the cylinder has height
$ht$ (both in miles). When the temperature is 2 degrees Fahrenheit and the humidity is 3
grams per cubic meter, what is the rate of change of $V$ the volume of the cylinder with
respect to temperature?

\[
	V = 2\text{(height)}\cdot\pi r^2 
\]
\[
	V = 2ht\pi(t + \frac{\sin(h)}{t})^2
\]
\[
	\frac{\partial V}{\partial t} = 2h\pi (t + \frac{\sin(h)}{t})^2 + 2ht\pi (2(t + \frac{\sin(h)}{t})(1 -\frac{\sin(h)}{t^2}))
\]
\[
	6\pi(2 + \frac{\sin(3)}{2})^2 + 24\pi(\frac{\sin(3)}{2})(1 - \frac{\sin(3)}{4}) = 85.94\text{ miles}^3/\text{Fahrenheit}
\]
\newpage
\section{Question 3}
Suppose $z$ is defined implicitly in terms of $x$ and $y$ via the equation $z^4 +z ^2 x^2 - y - 8 = 0$. Determine $\frac{\partial z}{\partial x} \text{ and } \frac{\partial z}{\partial y}$ at the point $(3, 2, 1)$.

\[
	\frac{\partial z}{\partial x} = \frac{-\frac{\partial F}{\partial x}}{\frac{\partial F}{\partial z}} \text{ where } F(x, y, z) = 0
\]

In our case, 
\[
	F(x, y, z) = z^4 + z^2x^2 - y -8
\]

\[
\frac{\partial F}{\partial x} = 2z^2x \quad \quad \frac{\partial F}{\partial z} = 4z^3 + 2zx^2 \quad \quad \frac{\partial F}{\partial y} = -1
\]
\[
	\text{At }(3, 2, 1), \frac{\partial F}{\partial x} = 6 \quad \quad \frac{\partial F}{\partial z} = 22
\]
\[
	\frac{\partial z}{\partial x} = -\frac{3}{11}
\]
\[
\frac{\partial z}{\partial y} = \frac{1}{22}
\]

\newpage
\section{Question 4}
Find the critical points of the function $e^{x^2-y^2+4y}$ and for each, use the second
derivative test to determine whether it is a local maximum, local minimum, saddle point,
or state that the second derivative test fails

Let $f(x, y)  = e^{x^2-y^2+4y}$

\[
	f_x = 2xe^{x^2-y^2+4y} \quad \quad f_y = (-2y+4) e^{x^2-y^2+4y}
\]
\[
	2xe^{x^2-y^2+4y} = 0 \quad x = 0 
\]
\[
(-2y+4) e^{x^2-y^2+4y} = 0 \quad y = -2
\]

There is a critical point at $(0, -2)$
\[
	D = f_{xx}(a, b) \cdot f_{yy}(a, b) - (f_{xy}(a, b))^2
\]
\[
f_{xx} = 2e^{x^2-y^2+4y} + 4x^2e^{x^2-y^2+4y}
\]
\[
	f_{yy} = -2e^{x^2-y^2+4y} + (-2y+4)^2e^{x^2-y^2+4y}
\]
\[
	f_{xy} = 2x(-2y+4)e^{x^2-y^2+4y}
\]
\[
	f_{xx} = e^{-8} \quad \quad f_{yy} = -2e^{-8} \quad \quad f_{xy} = 0
\]

\[
	D = e^{-8} \cdot -2e^{-8} + 0 = -2
\]

Since $D$ is less than 0, it is a saddle point 
\newpage
\section{Question 5}
Find the critical points of function $(x + y)\ln(x^2+y^2)$ and for each, use the second derivative test to determine whether it is a local maximum,minimum, saddle point, or state that the second derivative test fails


Let $f(x, y)  = (x + y)\ln(x^2+y^2)$ 

\[
	f_x = \ln(x^2+y^2)+ (x+y)\frac{2x^2}{x^2+y^2} \quad \quad f_{xx} = \frac{2x^{2}}{x^{2} + y^{2}} + \frac{4x \left(x + y\right)}{x^{2} + y^{2}} + \frac{2x}{x^{2} + y^{2}} - \frac{4x^{3} \left(x + y\right)}{\left(x^{2} + y^{2}\right)^{2}}
\]
\[
	f_y = \ln(x^2+y^2)+ (x+y)\frac{2y^2}{x^2+y^2} \quad \quad f_{y} = \frac{2y^{2}}{y^{2} + x^{2}} + \frac{4y \left(y + x\right)}{y^{2} + x^{2}} + \frac{2y}{y^{2} + x^{2}} - \frac{4y^{3} \left(y + x\right)}{\left(y^{2} + x^{2}\right)^{2}}
\]
\[
	f_{xy} = \frac{2x}{x^{2} + y^{2}} + \frac{2y^{2}}{x^{2} + y^{2}} - \frac{4y^{2} x \left(x + y\right)}{\left(x^{2} + y^{2}\right)^{2}}
\]

\[
	f_x = 0 \quad \ln(x^2+y^2)+ \frac{2x^2}{x^2+y^2} = 0
\]
\[
	f_y = 0 \quad \ln(x^2+y^2)+ \frac{2y^2}{x^2+y^2} = 0
\]
\[
	f_x = f_y \quad \ln(x^2+y^2)+ \frac{2x^2}{x^2+y^2} = \ln(x^2+y^2)+ \frac{2y^2}{x^2+y^2} 
\]
\[
	x = y
\]

\[
	\text{Let }f(x, y) = f(x, x)
\]
\[
	f(x, x) = (2x)\ln(2x^2)
\]
\[
	f_x(x, x) = 2\ln(2x^2) + 2x\frac{4x}{2x^2} = 0
\]
\[
	2\ln(2x^2) + 4 = 0
\]
\[
	\ln(2x^2) = -2
\]
\[
	2x^2 = \frac{1}{e^2}
\]
\[
	x^2 = \frac{1}{2e^2}
\]
\[
	x = \pm \frac{1}{\sqrt{2}e}
\]

The critical points occur at $(\frac{1}{\sqrt{2}e}, \frac{1}{\sqrt{2}e})$ and $(-\frac{1}{\sqrt{2}e}, -\frac{1}{\sqrt{2}e})$

\[
	D = f_{xx}(a, b) \cdot f_{yy}(a, b) - (f_{xy}(a, b))^2
\]
\[
	\text{At } (\frac{1}{\sqrt{2}e}, \frac{1}{\sqrt{2}e}) 
\]
\[
	f_{xx} = 6.84 \quad \quad f_{yy} = 6.84 \quad \quad f_{xy} = 2.844 
\]
\[
	D > 0 \text{ and } f_{xx} > 0 \text{, so point must be local minimum}
\]
\[
	\text{At } (-\frac{1}{\sqrt{2}e}, -\frac{1}{\sqrt{2}e}) 
\]
\[
	f_{xx} = -0.84 \quad \quad f_{yy} = -0.84 \quad \quad f_{xy} = -4.844 
\]
\[
	D < 0 \text{, so point must be saddle point}
\]
\newpage
\section{Question 6}
Find the maximum of $f(x, y) = y^2+xy-x^2$ on the square domain $0 \leq x \leq 2,
0 \leq y \leq 2$.

\[
	f_x = y - 2x \quad \quad f_{xx} = -2
\]
\[
	f_y = 2y + x \quad \quad f_{yy} = 2
\]
\[
	f_{xy} = 1
\]
\[
	f_x = 0 \quad y-2x = 0
\]
\[
	x = \frac{y}{2}
\]
\[
	y = - \frac{x}{2}
\]
\[
	x = 0 \quad \quad y = 0 
\]
At $(0, 0)$, 
\[
	D = (2 \cdot -2) -1  = -5
\]

Since $D < 0 $, the critical point is a saddle. The global maximum must exist on the boundary. The boundary is a square.

\[
B_1: x= 0 \quad B_2: x= 2 \quad B_3: y = 0 \quad B_4: y = 2 
\]

\[
	\text{At } B_1, \quad f(0, y)  = y^2
\]
\[
	\frac{d}{dy} y^2 = 2y \quad \text{Critical point} : (0, 0)
\]
\[
\frac{d^2}{dy^2} y^2 = 2  
\]
A potential minimum exists at $(0,0)$

\[
	\text{At } B_2, \quad f(2, y)  = y^2+2y - 4 
\]
\[
	\frac{d}{dy} y^2+2y-4 = 2y+2 \quad \text{Critical point} : (2, -1)
\]
Since the second derivative is positive, a potential minimum exists at $(2, -1)$

\[
	\text{At } B_3, \quad f(x, 0)  = -x^2 
\]
\[
	\frac{d}{dx} -x^2 = -2x \quad \text{Critical point} : (0, 0)
\]
\[
	\frac{d^2}{dx^2} -x^2 = -2
\]
A potential max exists at $(0, 0)$
\[
	\text{At } B_3, \quad f(x, 2)  = 4 + 2x -x^2  
\]
\[
	\frac{d}{dx} 4+2x-x^2 = 2-2x \quad \text{Critical point} : (1, 2)
\]
A potential max exists on $(1, 2)$
\\\\
Now, let's test all of the points. 

$f(0, 0) = 0 \quad f(2, -1) = -5 \quad f(1, 2) = 5$
Global minimum at $(2, -1)$ at $-5$, global maximum at $(1,2)$ at $5$

 


\newpage
\section{Question 7}
Find the maximum of $f(x, y) = xy(1-x-y)$ on the domain $D$ defined by $-1 \leq x \leq 1$ and $-1 \leq y \leq 1$. Find all critical points of $f$, and find the global maximum and global minimum for $f$ on $D$

\[
	f_x = y(1 - x -y) - xy \quad \quad f_{xx} = -y - y = -2y
\]
\[
	f_y = x(1-x-y) - xy \quad \quad f_{yy} = -x -x = -2x
\]
\[
	f_x = 0 \quad \quad y(1 - x -y) - xy = 0 
\]
\[
	y = 0 
\]

\[
	f_y = 0 \quad \quad x(1 - x - y) -xy = 0
\]
\[
	x = 0 
\]

A critical point occurs at $(0, 0)$

\[
	D = f_{xx}(a, b) \cdot f_{yy}(a, b) - f_{xy}(a, b)
\]
\[
	f_{xy} = (1-x-y) -y - x
\]
\[
	D = 0 \cdot 0 - 1 = 0
\]

$(0, 0)$ is a saddle point, let's test the boundary now

\[
	B_1: x = -1 \quad B_2: x = 1 \quad B_3: y = -1 \quad B_4: y = 1
\]
\[
	B_1: f(-1, y) = -y(2 - y) = -2y +y^2
\]
\[
	\frac{d}{dy}= -2y+ y^2 = -2 + 2y \quad \text{Potential min at (-1, 1)}
\]
\[
B_2: f(1, y) = y(- y) = -y^2
\]
\[
	\frac{d}{dy}= -y^2 = -2y \quad \text{Potential max at (1, 0)}
\]
\[
	B_3: f(x, -1) = -x(2 - x) = -2x +x^2
\]
\[
	\frac{d}{dx}= -2x+x^2 = -2 + 2x \quad \text{Potential min at (-1, -1)}
\]
\[
	B_4: f(x, 1) = x(-x) = -x^2
\]
\[
	\frac{d}{dx} =-x^2 = -2x \quad \text{Potential max at (0, 1)}
\]

Now let's test the points

\[
	f(-1, 1) = -1(1+1-1) = -1 \quad f(1, 0) = 0 \quad f(-1, -1) = 1(1 + 1 +1) = 3 \quad f(0, 1) = 0
\]
Global maximum occurs at $(-1, -1, 3)$, global mimimum occurs at $(-1, 1, -1)$

\end{document}

