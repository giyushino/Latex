\documentclass[hidelinks]{article}
\usepackage[a4paper, total={7in, 10in}, top=0.25in]{geometry}
\usepackage[dvipsnames]{xcolor}
\usepackage{amsmath}
\usepackage{tikz}
\usepackage{tkz-euclide}
\usepackage[unicode]{hyperref}
\usepackage[all]{hypcap}
\usepackage{fancyhdr}
\usepackage{amsfonts}
\usepackage[utf8]{inputenc}
\DeclareUnicodeCharacter{2212}{-}
\usepackage{amsmath}
\usepackage{array}
\mathchardef\mhyphen="2D % Define a "math hyphen"
\newcommand\rnumber{\mathop{r\mhyphen number}}

\usetikzlibrary{angles,calc, decorations.pathreplacing}

\definecolor{carminered}{rgb}{1.0, 0.0, 0.22}
\definecolor{capri}{rgb}{0.0, 0.75, 1.0}
\definecolor{brightlavender}{rgb}{0.75, 0.58, 0.89}
\title{\textbf{Math 33B Lecture 1}}
\author{Allan Zhang}
\date{March 31, 2025}
\begin{document}
\hypersetup{bookmarksnumbered=true,}
\definecolor{darkspringgreen}{rgb}{0.09, 0.45, 0.27}
\definecolor{darkseagreen}{rgb}{0.56, 0.74, 0.56}
\definecolor{green(munsell)}{rgb}{0.0, 0.66, 0.47}
\pagecolor{white}
\color{black}
\maketitle
\section*{Examples and Direction Fields}
\subsection*{Example 1: General Questions}
A car accelerates s.t. its velocity at time $t$ is $v(t) = e^t$ 
\vspace{0.2cm}

How far has the car traveled after $t$ seconds?

\textbf{Solution: }
\[
    \text{Let $x(t) = $ distance traveled at time $t$}
\]
\[
    \text{Then by definition, } \frac{\mathrm{d} x}{\mathrm{d} t} = v(t)
    \quad v(t) = e^t
\]
\[
    \text{Problems like this have a family of solutions: } e^t + C, \quad C \in \mathbb{R}
\]
\[
    \text{In this case, we must solve for $C$ which gives the correct solution}
\]
\[
    \text{We want } x(0) = 0 \quad x(0) = e^0 + C\quad 0 = 1 + C \quad C = -1 
\]
\[
    \text{More generally, for questions with the form }
    \frac{\mathrm{d} x}{\mathrm{d} t} = f(t) 
    \text{, we can solve by integrating}
\]
\[
    x(t) = \int^t f(S) \mathrm{d}s + C
\]
\subsection*{Example 2: Population Growth}
You go to a field and see $P_0$ rabbits and want to figure out how many rabbits in the future. 
To model rabbit population, need 2 ingredients: 
\begin{itemize}
    \item[] $\alpha$ = birth rate = probability given rabbit reproduces per unit time
    \item[] $\beta = $ death rate = probability given rabbit die in a unit time interval
\end{itemize}

\noindent
We want to find the population at time $t$, call this $P(t)$
\[
    P(t+\Delta t) - P(t)= \text{change in population}
\]
\[
    = \text{$\#$ born - $\#$ die}
\]
\[
    = \alpha \Delta t P(t) - \beta \Delta t P(t)
\]
\[\text{After dividing by $\Delta t $} \]
\[
    \frac{P(t + \Delta t) - P(t)}{\Delta t} = \alpha P(t) - \beta P(t) 
\]
\[
    \text{As $\Delta t \rightarrow 0$, } \frac{\mathrm{d}P}{\mathrm{d}t} = (\alpha - \beta) P(t)
\]
\[
    \text{When substituting }(\alpha - \beta) \text{ with $r$, we get } 
    \frac{\mathrm{d}P}{\mathrm{d}t} = rP(t) 
\]

Example solutions: $P(t) = e^{rt} \quad P(t) = 2e^{rt}$

Generally, solutions are of the form $Ce^{rt}$, where $C = P_0$, the initial population size $P(0)$
\newpage
\subsection*{Example 3: Logistic Growth}
A more realistic model for population growth is the logistic equation 
\[
    \frac{\mathrm{d}P}{\mathrm{d} t} = rP(1 - \frac{P}{K})
\]
where $K$ is the carrying capacity, the max population size the environment can hold
\[
    \text{If } P << K \rightarrow \frac{P}{K} \approx 0 \quad \frac{\mathrm{d}P}{\mathrm{d}t} = rP 
\]
     
\[
    \text{If } P \approx K \rightarrow \frac{P}{K} \approx 1 \quad 1 - \frac{P}{K} \rightarrow  
 \frac{\mathrm{d}P}{\mathrm{d}t} = 0
\]
\subsection*{Direction Fields}
Some equations can't be solved, but we can still learn about the solution from the direction field. 

    DEF: A DE is in normal form if it is written as 
    \[
         \frac{\mathrm{d}y}{\mathrm{d}x} = f(x, y)
    \]
\end{document}

\approx
\noindent

