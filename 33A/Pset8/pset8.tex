\documentclass[hidelinks]{article}
\usepackage[a4paper, total={7in, 10in}]{geometry}
\usepackage[dvipsnames]{xcolor}
\usepackage{amsmath}
\usepackage{tikz}
\usepackage{tkz-euclide}
\usepackage[unicode]{hyperref}
\usepackage[all]{hypcap}
\usepackage{fancyhdr}
\usepackage{amsfonts}
\usepackage[utf8]{inputenc}
\DeclareUnicodeCharacter{2212}{-}
\usepackage{amsmath}
\mathchardef\mhyphen="2D % Define a "math hyphen"
\newcommand\rnumber{\mathop{r\mhyphen number}}

\usetikzlibrary{angles,calc, decorations.pathreplacing}

\definecolor{carminered}{rgb}{1.0, 0.0, 0.22}
\definecolor{capri}{rgb}{0.0, 0.75, 1.0}
\definecolor{brightlavender}{rgb}{0.75, 0.58, 0.89}
\title{\textbf{MATH 33A Problem Set 8}}
\author{Allan Zhang}
\date{November 18th, 2024}
\begin{document}
\hypersetup{bookmarksnumbered=true,}
\definecolor{darkspringgreen}{rgb}{0.09, 0.45, 0.27}
\definecolor{darkseagreen}{rgb}{0.56, 0.74, 0.56}
\definecolor{green(munsell)}{rgb}{0.0, 0.66, 0.47}
\pagecolor{white}
\color{black}
\maketitle

\section{Question 1}
Find an eigenbasis for the matrix $A = \begin{bmatrix} 2 & 1 \\ 6 & 3 \end{bmatrix}$ 

\[
\text{det}(A - \lambda I) = \text{det}(\begin{bmatrix} 2 & 1 \\ 6 & 3 \end{bmatrix} - \lambda \begin{bmatrix} 1 & 0 \\ 0 & 1 \end{bmatrix}) = \text{det} (\begin{bmatrix} 2 - \lambda & 1 \\ 6 & 3 - \lambda \end{bmatrix})
\]

\[
(2 - \lambda)(3 - \lambda) - 6 = 0 
\]
\[
6 - 5\lambda + \lambda^2 - 6 = 0
\]
\[
\lambda(\lambda - 5) = 0
\] 
\[
\lambda = 0 \quad \quad \lambda = 5
\]
\\
\[
E_{\lambda = 0} = \text{ker}\begin{bmatrix} 2 & 1 \\ 6 & 3 \end{bmatrix} 
\]

\[
\begin{bmatrix} 2 & 1 \\ 6 & 3 \end{bmatrix} \begin{bmatrix} x \\ y \end{bmatrix} = \begin{bmatrix} 0 \\ 0 \end{bmatrix}
\]

\[
\begin{bmatrix}2 & 1 & 0\\ 6 & 3 & 0\end{bmatrix} = \begin{bmatrix}2 & 1 & 0\\ 0 & 0 & 0\end{bmatrix} = \begin{bmatrix}1 & 1/2 & 0\\ 0 & 0 & 0\end{bmatrix}
\]

\[
x = -1/2y
\]

\[
	\text{ker}(\begin{bmatrix} 2 & 1 \\ 6 & 3 \end{bmatrix}) = \{ y \begin{bmatrix} -1/2 \\ 1 \end{bmatrix}: y \in \mathbb{R}\}
\]
\\
\[
E_{\lambda = 5} = \text{ker}\begin{bmatrix} -3 & 1 \\ 6 & -2 \end{bmatrix} 
\]

\[
\begin{bmatrix} -3 & 1 \\ 6 & -2 \end{bmatrix} \begin{bmatrix} x \\ y \end{bmatrix} = \begin{bmatrix} 0 \\ 0 \end{bmatrix}
\]

\[
\begin{bmatrix}-3 & 1 & 0\\ 0 & 0 & 0\end{bmatrix}= \begin{bmatrix}1 & -1/2 & 0\\ 0 & 0 & 0\end{bmatrix}
\]
\[
	x = 1/3y
\]

\[
	\text{ker}(\begin{bmatrix} -3 & 1 \\ 6 & -2 \end{bmatrix}) = \{ y \begin{bmatrix} 1/3 \\ 1 \end{bmatrix}: y \in \mathbb{R}\}
\]
\\
\[
	\text{The eigenbasis is: } \{  \{ y \begin{bmatrix} -1/2 \\ 1 \end{bmatrix},  \begin{bmatrix} 1/3 \\ 1 \end{bmatrix}\}
\]

\newpage

\section{Question 2}
For each matrix below, find an invertible matrix $S$ and a diagonal matrix $D$ such that $S^{-1}AS = D$
\[
	A = \begin{bmatrix} 2 & 1 \\ 6 & 3 \end{bmatrix}

\]
\[
	\text{Remember that det}(A - \lambda I) = 0
\]
\[
	\text{det}(\begin{bmatrix} 2 - \lambda & 1 \\ 6 & 3 - \lambda \end{bmatrix}) = 0
\]
\[
	(2 - \lambda)(3 - \lambda) - 6 = 0
\]
\[
	6 - 5\lambda + \lambda^2 - 6 = 0 
\]
\[
	\lambda(\lambda - 5) = 0
\]
\[
	\lambda = 0 \quad \quad \lambda = 5
\]
\[
	\text{ker}(M - \lambda I) = \text{ker}(\begin{bmatrix} 2 & 1 \\ 6 & 3 \end{bmatrix} - 5\begin{bmatrix} 1 & 0 \\ 0 & 1 \end{bmatrix}) = \text{ker}(\begin{bmatrix} -3 & 1 \\ 6 & -2 \end{bmatrix})
\]
\[ 
	\begin{bmatrix} -3 & 1 & 0 \\ 6 & -2 & 0 \end{bmatrix} = \begin{bmatrix} 1 & -3 & 0  \\ 0 & 0 & 0 \end{bmatrix}
\]
\[
	\text{ker}(\begin{bmatrix} -3 & 1 \\ 6 & -2 \end{bmatrix}) = \{ y \begin{bmatrix} 1 \\ 3\end{bmatrix} : y \in \mathbb{R} \}	
\]

\[
	\text{ker}(M - \lambda I) = \text{ker}(\begin{bmatrix} 2 & 1 \\ 6 & 3 \end{bmatrix} - 0\begin{bmatrix} 1 & 0 \\ 0 & 1 \end{bmatrix}) = \text{ker}(\begin{bmatrix} 2 & 1 \\ 6 & 3 \end{bmatrix})
\]
\[
	\begin{bmatrix} 2 & 1 & 0 \\ 6 & 3 & 0 \end{bmatrix} = \begin{bmatrix} 1 & 1/2 & 0 \\ 0 & 0 & 0 \end{bmatrix}
\]

\[
	\text{ker}(\begin{bmatrix} 2 & 1 \\ 6 & 3 \end{bmatrix}) = \{ y \begin{bmatrix} 1 \\ -1/2\end{bmatrix} : y \in \mathbb{R} \}	
\]
\[
	S = \begin{bmatrix} 1 & 1 \\ 3 & -1/2 \end{bmatrix} \quad \quad D = \begin{bmatrix} 0 & 0 \\ 0 & 5 \end{bmatrix}
\]
\\
\[
	A = \begin{bmatrix} 3 & 6 \\ 7 & 2 \end{bmatrix}
\]
\[
	\text{det}(\begin{bmatrix} 3 & 6 \\ 7 & 2 \end{bmatrix} - \lambda I) = \text{det}(\begin{bmatrix} 3 - \lambda & 6 \\ 7 & 2 - \lambda \end{bmatrix}) 
\]
\[
	(3- \lambda)(2 - \lambda) - 42 = 0
\]
\[
	6 - 5\lambda + \lambda^2 - 42 = 0
\]
\[
	\lambda^2 -5\lambda - 36 = 0
\]
\[
	(\lambda - 9)(\lambda + 4) = 0 
\]
\[
	\lambda = 9 \quad \quad \lambda = -4
\]
\[
	\text{ker}(M - \lambda I) = \text{ker}(\begin{bmatrix} 3 & 6 \\ 7 & 2 \end{bmatrix} - 9\begin{bmatrix} 1 & 0 \\ 0 & 1 \end{bmatrix}) = \text{ker}(\begin{bmatrix} -6 & 6 \\ 7 & -7 \end{bmatrix})
\]
\[
	\begin{bmatrix} -6 & 6 \\ 7 & -7 \end{bmatrix}= \begin{bmatrix} 1 & -1 & 0\\ 7 & -7 & 0 \end{bmatrix} =   \begin{bmatrix} 1 & -1 & 0\\ 0 & 0 & 0 \end{bmatrix} 
\]
\[
	\{ y\begin{bmatrix} 1 \\ 1 \end{bmatrix} : y\in \mathbb{R}\}
\]
\[
	\text{ker}(M - \lambda I) = \text{ker}(\begin{bmatrix} 3 & 6 \\ 7 & 2 \end{bmatrix} - -4\begin{bmatrix} 1 & 0 \\ 0 & 1 \end{bmatrix}) = \text{ker}(\begin{bmatrix} 7 & 6 \\ 7 & 6 \end{bmatrix})
\]
\[
	\begin{bmatrix} 7 & 6 \\ 7 & 6 \end{bmatrix}= \begin{bmatrix} 1 & 6/7 & 0\\ 0 & 0 & 0 \end{bmatrix}\]
\[
	\{ y\begin{bmatrix} 1 \\ -6/7 \end{bmatrix} : y\in \mathbb{R}\}
\]

\[
	S = \begin{bmatrix} 1 & 1 \\ 1 & -6/7 \end{bmatrix} \quad \quad D = \begin{bmatrix} 9 & 0 \\ 0 & -4\end{bmatrix}	
\]
\\
\[
	A = \begin{bmatrix} 1 & 0 & 0 \\ -5 & 0 & 2 \\ 0 & 0 & 1 \end{bmatrix}
\]
\[
	\text{det}(A - \lambda I) = \text{ker}(\begin{bmatrix} 1 - \lambda & 0 & 0 \\ -5 & - \lambda & 2 \\ 0 & 0 & 1 -\lambda \end{bmatrix})
\]
\[
	= (1-\lambda)((-\lambda)(1-\lambda))
\]
\[
	(1 - \lambda)(\lambda^2 - \lambda)
\]
\[
	\lambda(1 - \lambda)(\lambda - 1)
\]
\[
	\lambda = - 1\quad \quad \lambda = 1
\]
\\
\[
	\text{ker}(\begin{bmatrix} 1 & 0 & 0 \\ -5 & 0 & 2 \\ 0 & 0 & 1 \end{bmatrix} - \lambda I)
\]
\[
	\text{ker}(\begin{bmatrix} 1 & 0 & 0 \\ -5 & 0 & 2 \\ 0 & 0 & 1 \end{bmatrix} - 1I) = 
	\text{ker}(\begin{bmatrix} 0 & 0 & 0 \\ -5 & -1 & 2 \\ 0 & 0 & 0 \end{bmatrix})
\]
\[
	-5x - y + 2z = 0 
\]
\[
	y  = -5x + 2z
\]
\[
	\text{The kernel is span}(\{\begin{bmatrix} 1 \\ -5 \\ 0 \end{bmatrix}, \begin{bmatrix} 0 \\ 2 \\ 1 \end{bmatrix}\})
\]
\\
\[
	\text{ker}(\begin{bmatrix} 1 & 0 & 0 \\ -5 & 0 & 2 \\ 0 & 0 & 1 \end{bmatrix} - (-1)I) =  
	\text{ker}(\begin{bmatrix} 2 & 0 & 0 \\ -5 & 1 & 2 \\ 0 & 0 & 2 \end{bmatrix} 
\]
\[
	\begin{bmatrix} 2 & 0 & 0 & 0 \\ -5 & 1 & 2 & 0 \\ 0 & 0 & 2 & 0\end{bmatrix} \rightarrow \begin{bmatrix} 2 & 0 & 0 & 0 \\ -5 & 1 & 2 & 0 \\ 0 & 0 & 2 & 0\end{bmatrix} \rightarrow \begin{bmatrix} 1 & 0 & 0 & 0 \\ -5 & 1 & 2 & 0 \\ 0 & 0 & 1 & 0\end{bmatrix} = \begin{bmatrix} 1 & 0 & 0 & 0 \\ 0 & 1 & 0 & 0 \\ 0 & 0 & 1 & 0\end{bmatrix}
\]
\[
	\begin{bmatrix}0 \\ 0 \\ 0\end{bmatrix}
\]
\[
	S = \begin{bmatrix} 1 & 0 & 0 \\ -5 & 2 & 0 \\ 0 & 1 & 0 \end{bmatrix} \quad \quad D = \begin{bmatrix} 1 & 0 & 0 \\ 0  & 1 & 0 \\ 0 &0 & 0\end{bmatrix}
\]
\newpage
\section{Question 3}
For what values of $a, b, c$ does the matrix 
\[
	A = \begin{bmatrix} a & b \\ c & d\end{bmatrix}
\]
have two distinct real eigenvalues

\[
	\text{det}(A - \lambda I) = \text{det}( \begin{bmatrix} a -\lambda & b \\ c & d - \lambda \end{bmatrix})
\]
\[
	(a - \lambda)(d - \lambda) - bc = 0
\]
\[
	(ad -a\lambda - d\lambda + \lambda^2) - bc = 0
\]
To find out how many roots, in this case $\lambda$ this equation has, we can use the discriminant. Discriminant = $p^2 - 4q$
\[
	p = -a - d \quad \quad q = ad - bc 
\]
\[
	\text{If } (- a - d)^2 -4(ad - bc) > 0 \text{, then there are two real roots, meaning there are two distinct real eigenvalues}
\]
\newpage
\section{Question 4}
For what values of $a, b, c$ are the matricies below diagonizable?
\[
	A = \begin{bmatrix} 1 & a & b \\ 0 & 1 & c \\ 0 & 0 & 1\end{bmatrix}
\]
\[
	\text{det}(\begin{bmatrix} 1 & a & b \\ 0 & 1 & c \\ 0 & 0 & 1\end{bmatrix} - \lambda I) =
	\text{det}(\begin{bmatrix} 1 -\lambda& a & b \\ 0 & 1 -\lambda & c \\ 0 & 0 & 1 -\lambda\end{bmatrix}) = (1- \lambda)^3 
\]
\[
	\lambda = 1 \text{ with multiplicity 3}
\]
\[
	\text{ker}(\begin{bmatrix} 1 & a & b \\ 0 & 1 & c \\ 0 & 0 & 1\end{bmatrix} - I) = \text{ker}(\begin{bmatrix} 0 & a & b \\ 0 & 0 & c \\ 0 & 0 & 0\end{bmatrix}) 
\]
\[
	\begin{bmatrix} 0 & a & b & 0 \\ 0 & 0 & c & 0 \\ 0 & 0 & 0 & 0\end{bmatrix} 
\]
\begin{align*}
	0x_0 + ax_1 + bx_2 = 0 \\
	cx_2 = 0
\end{align*}
As long as $a = 0$ and $c = 0$, we can make 3 linearly independent eigenvectors, meaning the matrix can be diagonal
\\
\\
\[
	A = \begin{bmatrix} 1 & a & b \\ 0 & 2 & c \\ 0 & 0 & 3\end{bmatrix}
\]
\[
	\text{det}(\begin{bmatrix} 1 & a & b \\ 0 & 1 & c \\ 0 & 0 & 1\end{bmatrix} - \lambda I) =
	\text{det}(\begin{bmatrix} 1 -\lambda& a & b \\ 0 & 2 -\lambda & c \\ 0 & 0 & 3 -\lambda\end{bmatrix}) = (1- \lambda)(2 - \lambda)(3 - \lambda) 
\]
\[
	\lambda = 1 \quad \quad \lambda = 2 \quad \quad \lambda = 3
\]
\[
	\text{ker}(\begin{bmatrix} 1 & a & b \\ 0 & 2 & c \\ 0 & 0 & 3\end{bmatrix} - I) = \text{ker}(\begin{bmatrix} 0 & a & b \\ 0 & 1 & c \\ 0 & 0 & 2\end{bmatrix}) 
\]
\[
	\begin{bmatrix} 0 & a & b & 0 \\ 0 & 1 & c & 0\\ 0 & 0 & 2 & 0\end{bmatrix}
\]
\begin{align*}
	ay + bz = 0 \\
	y + cz = 0 \\
	2z = 0 \\
	\\
	y= 0 \quad z = 0 \\
	\{x \begin{bmatrix} 1 \\ 0 \\  0 \end{bmatrix} : x \in \mathbb{R}\}
\end{align*}

\[
	\text{ker}(\begin{bmatrix} 1 & a & b \\ 0 & 2 & c \\ 0 & 0 & 3\end{bmatrix} - 2I) = \text{ker}(\begin{bmatrix} -1 & a & b \\ 0 & 0 & c \\ 0 & 0 & 1\end{bmatrix}) 
\]
\[
	\begin{bmatrix} -1 & a & b & 0 \\ 0 & 0 & c & 0\\ 0 & 0 & 1 & 0\end{bmatrix}
\]
\begin{align*}
	-x + ay +bz = 0 \\
	cz = 0 \\
	1z = 0 \\
	\{y \begin{bmatrix} a \\ 1 \\  0 \end{bmatrix} : y \in \mathbb{R}\}
\end{align*}

\[
	\text{ker}(\begin{bmatrix} 1 & a & b \\ 0 & 2 & c \\ 0 & 0 & 3\end{bmatrix} - 3I) = \text{ker}(\begin{bmatrix} -2 & a & b \\ 0 & -1 & c \\ 0 & 0 & 0\end{bmatrix}) 
\]
\[
	\begin{bmatrix} -2 & a & b & 0 \\ 0 & -1 & c & 0\\ 0 & 0 & 0 & 0\end{bmatrix}
\]
\begin{align*}
	-2x + ay +bz = 0 \\
	-y + cz = 0 \\\\
	y = cz \\
	-2x + acz + bz = 0 \\
	x = z\frac{ac + b}{-2} \\
	\{z \begin{bmatrix} \frac{ac + b}{-2} \\ c \\  1 \end{bmatrix} : z \in \mathbb{R}\}
\end{align*}

For any value of $a, b, c$, since there are 3 unique eigenvectors, the matrix will be diagonal


\end{document}
