\documentclass[hidelinks]{article}
\usepackage[a4paper, total={7in, 10in}]{geometry}
\usepackage[dvipsnames]{xcolor}
\usepackage{amsmath}
\usepackage{tikz}
\usepackage{tkz-euclide}
\usepackage[unicode]{hyperref}
\usepackage[all]{hypcap}
\usepackage{fancyhdr}
\usepackage{amsfonts}
\usepackage[utf8]{inputenc}
\DeclareUnicodeCharacter{2212}{-}
\usepackage{amsmath}
\usepackage{array}
\mathchardef\mhyphen="2D % Define a "math hyphen"
\newcommand\rnumber{\mathop{r\mhyphen number}}

\usetikzlibrary{angles,calc, decorations.pathreplacing}

\definecolor{carminered}{rgb}{1.0, 0.0, 0.22}
\definecolor{capri}{rgb}{0.0, 0.75, 1.0}
\definecolor{brightlavender}{rgb}{0.75, 0.58, 0.89}
\title{\textbf{Integration Review and Other Important Concepts for 32B}}
\author{Allan Zhang}
\date{Doesn't matter}
\begin{document}
\hypersetup{bookmarksnumbered=true,}
\definecolor{darkspringgreen}{rgb}{0.09, 0.45, 0.27}
\definecolor{darkseagreen}{rgb}{0.56, 0.74, 0.56}
\definecolor{green(munsell)}{rgb}{0.0, 0.66, 0.47}
\pagecolor{white}
\color{black}
\maketitle
\section{Basic Integration Rules}

\[
	\text{Power Rule:}	\int x^n \, dx = \frac{x^{n+1}}{n+1} + C
\]
\[
	\text{E Rule?:} \int e^x \, dx = e^x + C 
\]
\[
	\text{}
\]

\end{document}

