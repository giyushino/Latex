\documentclass[hidelinks]{article}
\usepackage[a4paper, total={7in, 10in}]{geometry}
\usepackage[dvipsnames]{xcolor}
\usepackage{amsmath}
\usepackage{tikz}
\usepackage{tkz-euclide}
\usepackage[unicode]{hyperref}
\usepackage[all]{hypcap}
\usepackage{fancyhdr}
\usepackage{amsfonts}
\usepackage[utf8]{inputenc}
\DeclareUnicodeCharacter{2212}{-}
\usepackage{amsmath}
\usepackage{array}
\mathchardef\mhyphen="2D % Define a "math hyphen"
\newcommand\rnumber{\mathop{r\mhyphen number}}

\usetikzlibrary{angles,calc, decorations.pathreplacing}

\definecolor{carminered}{rgb}{1.0, 0.0, 0.22}
\definecolor{capri}{rgb}{0.0, 0.75, 1.0}
\definecolor{brightlavender}{rgb}{0.75, 0.58, 0.89}
\title{\textbf{MATH 32B Problem Set 1}}
\author{Allan Zhang}
\date{January 6, 2025}
\begin{document}
\hypersetup{bookmarksnumbered=true,}
\definecolor{darkspringgreen}{rgb}{0.09, 0.45, 0.27}
\definecolor{darkseagreen}{rgb}{0.56, 0.74, 0.56}
\definecolor{green(munsell)}{rgb}{0.0, 0.66, 0.47}
\pagecolor{white}
\color{black}
\maketitle

\section{Question 11}
The following table gives the approximate height at quarter-meter intervals of a mound of gravel. Estimate the volume of the mound by computing the average of the two Riemann sums $S_{4,3}$ with lower-left and upper-right vertices of the subrectangles as sample points.

\begin{center}
\renewcommand{\arraystretch}{1.5} % Adjust row height
\setlength{\tabcolsep}{10pt} % Adjust column spacing
\begin{tabular}{|c|c|c|c|c|c|c|}
\hline
\textbf{0.75} & 0.1  & 0.2  & 0.2  & 0.15 & 0.1  \\ \hline
 \textbf{0.5} & 0.2  & 0.3  & 0.5  & 0.4  & 0.2  \\ \hline
 \textbf{0.25}& 0.15 & 0.2  & 0.4  & 0.3  & 0.2  \\ \hline
 \textbf{$y:$ 0}    & 0.1  & 0.15 & 0.2  & 0.15 & 0.1  \\ \hline
& \textbf{$x:$ 0} & \textbf{0.25} & \textbf{0.5}  & \textbf{0.75} & \textbf{1}    \\ \hline
\end{tabular}
\end{center}

Computing area using lower-left verticies as sample points:
The verticies we will be using are: 
\[\begin{bmatrix}
	0.2 & 0.3 & 0.5 & 0.4 \\ 
	0.15 & 0.2 & 0.4  & 0.3\\
	0.1 & 0.15 & 0.2  & 0.15
\end{bmatrix}\]

Note that $\Delta x$ and $\Delta y$ are both 0.25, meaning the volume of each rectangle will be the height of the rectangles times $0.25^2$, which is 0.0625
\[ 0.0625 \cdot \begin{bmatrix}
	0.2 & 0.3 & 0.5 & 0.4 \\ 
	0.15 & 0.2 & 0.4  & 0.3\\
	0.1 & 0.15 & 0.2  & 0.15
\end{bmatrix} = \begin{bmatrix}
	0.0125 & 0.01875 & 0.03125 & 0.025 \\ 
	0.009375 & 0.0125 & 0.025  & 0.01875\\
	0.00625 & 0.009375 & 0.0125  & 0.009375
\end{bmatrix} \]

Adding up these areas gives us 0.190625

Now, let's repeat, except we will use the upper-right verticies as sample points

\[ 0.0625 \cdot \begin{bmatrix}
	0.2  & 0.2  & 0.15 & 0.1 \\ 
	0.3  & 0.5  & 0.4  & 0.2 \\
	0.2  & 0.4  & 0.3  & 0.2
\end{bmatrix} = 
\begin{bmatrix}
	0.0125 & 0.0125 & 0.009375 & 0.00625 \\ 
	0.01875 & 0.03125 & 0.025 & 0.0125 \\
	0.0125 & 0.025 & 0.01875 & 0.03125
\end{bmatrix} \]

Adding up these areas gives us 0.196875. The average of these two sums is \textbf{0.19375}, which is our estimate for the volume of the mound.
\newpage
\section{Question 15}
Use symmetry to evauate the double integral
\[
	\iint_\mathcal{R} x^3 \,dA, \quad \mathcal{R} = [-4, 4] \times [0, 5]
\]

Note that that $x^3$ is odd and lacks horizontal or vertical shifts. This means so for some value $a$, the area between the curve and $[-a, 0]$ will be equal to the area between the curve and $[0, a]$ multiplied by $-1$.

Also, note that we can split the domain into two sub-domains. Let us split $\mathcal{R}$ into $\mathcal{R}_1 = [-4, 0] \times [0, 5]$ and $\mathcal{R}_2 = [0, 4] \times [0, 5]$.
\[
	\iint_\mathcal{R} x^3 \,dA, \quad \mathcal{R} = [-4, 4] \times [0, 5] = 
\iint_{\mathcal{R}_1} x^3 \,dA, \quad \mathcal{R}_1 = [-4, 0] \times [0, 5] + 
	\iint_{\mathcal{R}_2} x^3 \,dA, \quad \mathcal{R}_2 = [0, 4] \times [0, 5]
\]

From the observation we stated before, it can be said that 
\[
\iint_{\mathcal{R}_1} x^3 \,dA, \quad \mathcal{R}_1 = [-4, 0] \times [0, 5] =  
	-\iint_{\mathcal{R}_2} x^3 \,dA, \quad \mathcal{R}_2 = [0, 4] \times [0, 5]
\]

We can rewrite the original expression as 
\[
-\iint_{\mathcal{R}_2} x^3 \,dA + 
	\iint_{\mathcal{R}_2} x^3 \,dA = 0 
\]

This double integral evaluates to 0.
\newpage
\section{Question 17}
Use symmetry to evaluate the double integral
\[
	\iint_\mathcal{R} \sin{x} \,dA, \quad \mathcal{R} = [0, 2\pi] \times [0, 2\pi]
\]

Note that $\sin{x}$ oscillates from -1 to 1. Importantly, without any transformations, $\sin x$ completes one whole revolution from 
$0 \text{ to } 2\pi$, and the function from $0 \text{ to } \pi$ is identical to the function from $\pi \text{ to } 2\pi$ reflected over the $x-$axis. 

Also remember that we can split the domain into two sub-domains. Let us split $\mathcal{R}$ into $\mathcal{R}_1 = [0, \pi] \times [0, 2\pi]$ and $\mathcal{R}_2 = [\pi, 2\pi] \times [0, 2\pi]$.

\[
	\iint_\mathcal{R} \sin{x} \,dA, \quad \mathcal{R} = [0, 2\pi] \times [0, 2\pi] =
	\iint_{\mathcal{R}_1} \sin{x} \,dA, \quad \mathcal{R}_1 = [0, \pi] \times [0, 2\pi] +
	\iint_{\mathcal{R}_2} \sin{x} \,dA, \quad \mathcal{R}_2 = [\pi, 2\pi] \times [0, 2\pi]
\]

From what we said before, we can observe that
\[
	\iint_{\mathcal{R}_1} \sin{x} \,dA, \quad \mathcal{R}_1 = [0, \pi] \times [0, 2\pi] = -
	\iint_{\mathcal{R}_2} \sin{x} \,dA, \quad \mathcal{R}_2 = [\pi, 2\pi] \times [0, 2\pi]
\]

Finally, we can state that
\[
	\iint_\mathcal{R} \sin{x} \,dA, \quad \mathcal{R} = [0, 2\pi] \times [0, 2\pi] =
	-\iint_{\mathcal{R}_2} \sin{x} \,dA + \iint_{\mathcal{R}_2} \sin{x} \,dA = 0
\]

The double integral evaluates to 0.
\newpage
\section{Question 19}
Evaluate the iterated integral
\[
	\int_1^3 \int_0^ 2 x^3y \,dy \,dx
\]

\[
	\int_1^3 \int_0^ 2 x^3y \,dy \,dx =\int_1^3 \left(\int_0^ 2 x^3y \,dy \right) dx 
\]
\[
	= \int_1^3 \left( \left. x^3\cdot \frac{y^2}{2}\right|_{y=0}^{y=2} \right) dx 
\]
\[
	= \int_1^3 \left( x^3\cdot (\frac{4}{2} - \frac{0}{2})  \right) dx 
\]
\[
	= \int_1^3 2x^3 \,dx = \left. \frac{x^4}{2} \right |_{x = 1}^{x=3}  = \frac{81}{2} - \frac{1}{2} = \frac{80}{2} = 40
\]

The iterated integral evaluates to 40.
\newpage
\section{Question 21}
Evaluate the iterated integral
\[
	\int_4^9 \int_{-3}^{8} 1 \,dx \,dy
\]
\[
	\int_4^9 \left ( \int_{-3}^{8} 1 \,dx \right )dy = 
\]
\[
	 \int_4^9 x \Big|_{x=-3}^{x=8} dy = 
\]
\[
	\int_4^9 11 \, dy =  11x \Big |_{x=4}^{x=9} = 99 - 44 = 55 
\]

The iterated integral evaluates to 55.

\newpage
\section{Question 23}
Evaluate the iterated integral
\[
	\int_{-1}^1 \int_{0}^{\pi} x^2 \sin y \,dy \,dx
\]
\[	
	\int_{-1}^1 \left (\int_{0}^{\pi} x^2 \sin y \,dy \right )dx
\]
\[
	\int_{-1}^1 \left( x^2 \cdot -\cos y  \Big |_{y = 0}^{y=\pi} \right )dx
\]
\[
	\int_{-1}^1  x^2 (1 - (-1)) \, dx
\]
\[
	\int_{-1}^1  2x^2 \, dx = \left . \frac{2x^3}{3} \right |_{x=-1}^{x=1} = \frac{2}{3} - \frac{-2}{3} = \frac{4}{3}
\]

The iterated integral evaluates to $\frac{4}{3}$.
\newpage
\section{Question 25}
Evaluate the iterated integral
\[
\int_2^6 \int_1^4 x^2\, dx\,dy
\]
\[
\int_2^6 \left( \int_1^4 x^2\, dx \right) dy
\]
\[
	\int_2^6 \left( \left. \frac{x^3}{3} \right|_{x=1}^{x=4} \right) dy
\]
\[
	\int_2^6  \frac{64}{3} - \frac{1}{3} \,dy= 
	\int_2^6  \frac{63}{3} \,dy
\]
\[
	\left . \frac{63x}{3} \right|_{x=2}^{x=6} = \frac{378 - 126}{3} = \frac{252}{3} = 84
\]

The iterated integral evaluates to 84.
\newpage
\section{Question 37}
Evaluate the integral
\[
	\iint_{\mathcal{R}}\frac{x}{y} \,dA, \quad \mathcal{R} = [-2, 4] \times [1, 3]
\]
\[
	\iint_{\mathcal{R}}\frac{x}{y} \,dx \,dy, \quad \mathcal{R} = [-2, 4] \times [1, 3]
\]
\[
	\int_{1}^{3}\int_{-2}^4\frac{x}{y} \,dx \,dy
\]
\[
	\int_{1}^{3} \left ( \left . \frac{x^2}{2y} \right |_{x = -2}^{x=4}\,dx \right)dy
\]
\[
	\int_{1}^{3} \frac{1}{2y} (16 - 4)\,dy
\]
\[
	\int_{1}^{3} \frac{6}{y}\,dy
\]
\[
	6\ln{y} \Big |_{y=1}^{y=3} = 6\ln{3} - 6\ln{1} = 6\ln{3}
\]
\newpage
\section{Question 40}
Evaluate the integral
\[
	\iint_{\mathcal{R}}\frac{y}{x + 1} \,dA, \quad \mathcal{R} = [0, 2] \times [0, 4] 
\]
\[
	\int_0^4 \int_0^2 \frac{y}{x+1} \,dx\,dy 
\]
\[
	\int_0^4 \int_0^2 \frac{y}{x+1} \,dx\,dy  
\]
\[
	\int_0^4 \left (y \ln|x+1| \Big|_{x=0}^{x=2} \right ) dy
\]
\[
	\int_0^4 y( \ln|3| - \ln|1|) \, dy
\]
\[
	\int_0^4 y \ln|3|\,  d
\]
\[
	\frac{\ln|3|\,y^2}{2} \Big |_{y = 0}^{y = 4}
\]
\[
	8\ln3 
\]

The solution to this evaluated integral is $8\ln 3$
\newpage
\section{Question 45}
Let $f(x, y) = mxy^2$, where $m$ is a constant. Find a value of $m$ such that $\iint_{\mathcal{R}} f(x, y) \, dA = 1$, where $\mathcal{R} = [0,1] \times [0,2]$

\[
	\int_0^2 \int_0^1 mxy^2 \,dx\,dy 
\]
\[
	\int_0^2 \left ( \left . \frac{mx^2y^2}{2} \right |_{x = 0}^{x = 1} \right )dy 
\]
\[
	\int_0^2 \frac{my^2}{2}\,dy 
\]
\[
	\left. \frac{my^2}{2} \right |_{y = 0}^{y = 2} 
\]
\[
	2m = 1
\]
\[
	m = \frac{1}{2}
\]

When $m = \frac{1}{2}$, the expression $\iint_{\mathcal{R}} f(x, y) \, dA, \mathcal{R} = [0,1] \times [0,2] = 1$ is true! 

\newpage
\section{Question 48}
a) Which is easier, antidifferentiating $xe^{xy}$ with respect to $x$ or with respect to $y$? Explain.\\
b) Evaluate $\iint_{\mathcal{R}} xe^{xy} \, dA$, where $\mathcal{R} = [0,1] \times [0,1]$

a) It is easier antidifferentiating $xe^{xy}$ with respect to $y$, as we can simply treat $x$ as a constant. Computing $\int xe^{xy} \, dy$ requires the same steps as solving $\int 4e^{4y}\,dy$, which is relatively simple

b)
\[
	\iint_{\mathcal{R}} xe^{xy} \, dA, \quad \mathcal{R} = [0,1] \times [0,1]
\]
\[
	\int_0^1 \int_0^1 xe^{yx} \,dy \, dx
\]
\[
	\int_0^1 \left(e^{yx} \Big |_{y = 0}^{y = 1}\right)dx
\]
\[
	\int_0^1 e^{x} - 1\,dx
\]
\[
	e^x - x \Big |_{x = 0}^{x = 1}
\]
\[
	e - 1 - 1 - 0 = e -2
\]

The integral evaluates to $e-2$


\newpage
\section{Question 49}
a) Which is easier, antidifferentiating $\frac{y}{1+xy}$ with respect to $x$ or with respect to $y$? Explain.\\
b) Evaluate $\iint_{\mathcal{R}} \frac{y}{1+xy} \, dA$, where $\mathcal{R} = [0,1] \times [0,1]$

a) It is easier to antidifferentiate $\frac{y}{1+xy}$ with respect to $x$, as we would only need to worry about one variable in the denominator. If we had integrated with respect to $y$, we would need to consider the $y$ in the numerator and the $y$ in the denominator. Integrating $\int \frac{2}{1+2x} dx$ can be solved with basic integration rules. However, antidifferentiating $\int \frac{y}{1+2y}$ would likely need to be solved using a technique like integration by parts. 

b)
\[
	\iint_{\mathcal{R}} \frac{y}{1+xy} \, dA, \quad \mathcal{R} = [0,1] \times [0,1]
\]
\[
	\int_0^1 \int_0^1 \frac{y}{1+xy} \,dx \, dy
\]
\[
	\int_0^1 \left(\left . \frac{y \ln|1+yx|}{y} \right |_{y = 0}^{y = 1}\right)dx
\]
\[
	\int_0^1 \ln|1+x| \,dx
\]
\[
	(1+x)\ln(1+x) - (1+x) \Big |_{x = 0 }{x = 1}
\]
\[
	2\ln2 - 2 - (0 - 1) = \ln2 -1
\]

This integral evaluates to $2\ln2 - 1$, or $\ln4 - 1$
\end{document}
