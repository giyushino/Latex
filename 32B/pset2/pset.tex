\documentclass[hidelinks]{article}
\usepackage[a4paper, total={7in, 10in}]{geometry}
\usepackage[dvipsnames]{xcolor}
\usepackage{amsmath}
\usepackage{tikz}
\usepackage{tkz-euclide}
\usepackage[unicode]{hyperref}
\usepackage[all]{hypcap}
\usepackage{fancyhdr}
\usepackage{amsfonts}
\usepackage[utf8]{inputenc}
\DeclareUnicodeCharacter{2212}{-}
\usepackage{amsmath}
\usepackage{array}
\mathchardef\mhyphen="2D % Define a "math hyphen"
\newcommand\rnumber{\mathop{r\mhyphen number}}

\usetikzlibrary{angles,calc, decorations.pathreplacing}

\definecolor{carminered}{rgb}{1.0, 0.0, 0.22}
\definecolor{capri}{rgb}{0.0, 0.75, 1.0}
\definecolor{brightlavender}{rgb}{0.75, 0.58, 0.89}
\title{\textbf{MATH 32B Problem Set 2}}
\author{Allan Zhang}
\date{January 15, 2025}
\begin{document}
\hypersetup{bookmarksnumbered=true,}
\definecolor{darkspringgreen}{rgb}{0.09, 0.45, 0.27}
\definecolor{darkseagreen}{rgb}{0.56, 0.74, 0.56}
\definecolor{green(munsell)}{rgb}{0.0, 0.66, 0.47}
\pagecolor{white}
\color{black}
\maketitle

\section{Question 5}
Compute the double integral of $f(x,y) = x^2y$ over the given shaded domain.
\[
    \text{$y$ is bounded by $y = 0$ and $y = 2$, $x$ is bounded by $x = 4-2y$ and $x = 4$}
\]
\[
    \int_{0}^2 \int_{4 -2y}^{4} x^2y \, dx \, dy
\]
\[
    \int_{0}^2 \left . \frac{x^3y}{3} \right |_{x = 4 -2y}^{x = 4}\, dy
\]
\[
    \int_{0}^2 \frac{64y}{3} - \frac{y(4-2y)^3}{3}\, dy
\]
\[
    \int_{0}^2 \frac{y}{3}(64 - (4-2y)^3) \, dy
\]
\[
    (4-2y)^3 = (16 - 16y + 4y^2)(4 - 2y) = 64 - 64y + 16y^2 - 32y + 32y^2 -8y^3 
\]
\[
    = 64 - 96y + 48y^2 - 8y^3   
\]
\[
    \int_{0}^2 \frac{y}{3}(8y^3 - 48y^2 + 96y) \, dy
\]
\[
    \int_{0}^2 \frac{8y^4}{3} - 16y^3 + 32y^2) \, dy
\]
\[
    \left. \frac{8y^5}{15} - 4y^4 + \frac{32y^3}{3} \right |_{y = 0}^{y = 2}
\]
\[
    \frac{256}{15} - 64 + \frac{256}{3} = \frac{256}{15} - \frac{960}{15} + \frac{1280}{15}
\]
The double integral computed over the given shaded domain is $\frac{192}{5}$


\newpage
\section{Question 8}
Sketch the domain $\mathcal{D}$ by $x+y \leq 12, x \geq 4, y \geq 4$ and compute $\iint_{\mathcal{D}}e^{x+y}dA$
\begin{center}
\begin{tikzpicture}[scale=0.4]

% Axes
\draw[->] (0,0) -- (14,0) node[right] {$x$};
\draw[->] (0,0) -- (0,14) node[above] {$y$};

% Lines for the inequalities
\draw[thick,blue] (0,12) -- (12,0) node[above right] {$x+y=12$}; % Line for x+y=12
\draw[thick,red] (4,0) -- (4,14) node[above right] {$x=4$}; % Line for x=4
\draw[thick,red] (0,4) -- (14,4) node[above right] {$y=4$}; % Line for y=4

% Fill the region
\fill[cyan,opacity=0.3] (4,4) -- (4,8) -- (8,4) -- cycle;

% Labels
\node at (4,4) [below left] {$(4,4)$};
\node at (4,8) [left] {$(4,8)$};
\node at (8,4) [below] {$(8,4)$};

\end{tikzpicture}
\end{center}
From this sketch, we can see that $y$ can be bounded by $y = 4$ and $y = 12 - x$. $x$ is bounded from $x = 4$ to $x = 8$
\[
    \iint_{\mathcal{D}}e^{x+y}dA = \int_{4}^8 \int_{4}^{12-x} e^{x+y} \, dy \,dx
\]
\[
    \int_{4}^8 \left ( e^{x+y} \Big|_{4}^{12-x}\right ) \,dx
\]
\[
    \int_{4}^8 e^{12} - e^{x+4} \,dx
\]
\[
    \left . e^{12}x - e^{x+4} \right |_{4}^8
\]
\[
    8e^12 - e^12 - (4e^12 -e^8) 
\]
The answer is $3e^{12} - e^8$

\newpage
\section{Question 12}
Calculate the double integral of $f(x,y) = y^2$ over the rhombus $\mathcal{R}$

Taking into account the symmetry of the rhombus and the symmetry of the $f(x,y) = y^2$, we can compute the volume of the curve over $\frac{1}{4}$ of the rhombus and multiply it by 4 to compute the total volume. 
We can define the bounds as: 
\[
    0 \leq x \leq 1, \quad 0 \leq y \leq 2-2x
\]
\[
    \int_{0}^1 \int_{0}^{2-2x} y^2 \, dy \, dx
\]
\[
    \int_{0}^1 \left. \frac{y^3}{3} \right |_{y = 0}^{y = 2-2x} \, dx
\]
\[
    (2-2x)^3 = 8 - 24x + 24x^2 - 8x^3
\]
\[
    \int_{0}^1 \frac{8 - 24x + 24x^2 - 8x^3}{3} \, dx
\]
\[
    \frac{1}{3}\int_{0}^1 8 - 24x + 24x^2 - 8x^3 \, dx
\]
\[
    \frac{1}{3} \Big [8x-12x^2 + 8x^3 - 2x^4 \Big ]_{0}^{1}
\]
\[
    \frac{1}{3} (8-12+8-2) = \frac{1}{3} (2) = \frac{2}{3}
\]
Since this is the volume computed over $\frac{1}{4}$ of the rhombus, the total volume is $4(\frac{2}{3}) = \frac{8}{3}$

\newpage

\section{Question 21}
$f(x,y) = 6xy-x^2$; bounded below by $y =x^2$, above by $y = \sqrt{x}$
The bounds are 
\[
    0 \leq x \leq 1, \quad x^2 \leq y \leq \sqrt{x}
\]
\[
    \int_{0}^1 \int_{x^2}^{\sqrt{x}} 6xy - x^2 \, dy \, dx
\]
\[
    \int_0^1 3xy^2 - x^2y \Big |_{y = x^2}^{y = \sqrt{x}} \, dx
\]
\[
    \int_0^1 3x(\sqrt{x})^2 - x^2(\sqrt{x}) - 3x(x^2)^2 + x^2(x^2) \, dx
\]
\[
    \int_0^1 3x^2 - x^{5/2} - 3x^5 + x^4 \, dx
\]
\[
    \left . x^3 - \frac{2x^{\frac{7}{2}}}{7} - \frac{x^6}{2} + \frac{x^5}{5} \right |_{0}^{1}
\]
\[
    1 - \frac{2}{7} - \frac{1}{2} + \frac{1}{5} = \frac{70}{70} - \frac{20}{70} - \frac{35}{70} + \frac{14}{70} = \frac{29}{70}
\]
The answer is $\frac{29}{70}$
\newpage
\section{Question 29}
Sketch the domain $\mathcal{D} \text{ corresponding to } \int_0^4 \int_{\sqrt{y}}^2 \sqrt{4x^2 + 5y} \, dx \, dy$. Then change the order of integration and evaluate. 

\begin{center}
\begin{tikzpicture}[scale=1.2]
    % Axes
    \draw[->] (-0.5,0) -- (3,0) node[right] {$x$};
    \draw[->] (0,-0.5) -- (0,5) node[above] {$y$};
    
    % Parabola x = sqrt(y)
    \draw[domain=0:4,smooth,variable=\y,blue,thick] plot ({sqrt(\y)},\y) node[above] {$x = \sqrt{y}$};
    
    % Vertical line x = 2
    \draw[red,thick] (2,0) -- (2,4);
    
    % Horizontal line y = 4
    \draw[dashed] (-0.5,4) -- (2.5,4) node[right] {$(2, 4)$};
    
    % Shaded region
    \fill[gray,opacity=0.3] 
        (0,0) 
        -- (2,0) 
        -- (2,4)
        -- plot[domain=4:0,smooth,variable=\y] ({sqrt(\y)},\y)
        -- cycle;
    
    % Intersection point
    \node[fill=black,circle,inner sep=1.5pt] at (2,4) {};
\end{tikzpicture}

\end{center}
After switching the bounds, we get:
\[
    0 \leq x \leq 2, \quad 0 \leq y \leq x^2
\]
\[
    \int_0^2 \int_0^{x^2} \sqrt{4x^2 + 5y} \, dy \, dx
\]
\[
    \int_0^2 \left( \left .\frac{2}{5\cdot3} (4x^2+5y)^\frac{3}{2} \right|_{y= 0}^{y=x^2} \right)
\]
\[ 
    \frac{2}{15}\int_0^2 (9x^2)^{\frac{3}{2}} - (4x^2)^{\frac{3}{2}} \, dx
\]
\[
    \frac{2}{15}\int_0^2 27x^3 - 8x^3 \, dx
\]
\[
    \frac{2}{15}\int_0^2 19x^3 \, dx
\]
\[
    \frac{2}{15} \Big[ \frac{19x^4}{4} \Big]_{x = 0}^{x = 2}
\]
\[
    \frac{2}{15} \cdot 76 = \frac{152}{15}
\]
The answer is $\frac{152}{15}$
\newpage
\section{Question 5}
$f(x, y, z) = (x-y)(y-z); \quad [0,1] \times [0,3] \times [0,3]$. 
\[
    \int_0^1 \int_0^3 \int_0^3 (x-y)(y-z) \,dz \, dy \, dx
\]
\[
    \int_0^1 \int_0^3 \int_0^3 xy-xz-y^2+yz \,dz \, dy \, dx
\]
\[
    \int_0^1 \int_0^3 \left . xyz-\frac{xz^2}{2}-y^2z+\frac{yz^2}{2} \right |_{z=0}^{z=3} dy \, dx
\]
\[
    \int_0^1 \int_0^3 3xy-\frac{9x}{2}-3y^2+\frac{9y}{2} \, dx
\]
\[
    \int_0^1 \left. \frac{3xy^2}{2}-\frac{9yx}{2}-y^3+\frac{9y^2}{4} \right|_{y=0}^{y=3} \, dx
\]
\[
    \int_0^1 \frac{27x}{2}-\frac{27x}{2}-27+\frac{81}{4} \, dx
\]
\[
    \left. \frac{27x^2}{4}-\frac{27x}{2}-27+\frac{81}{4} \right |_{x = 0}^{x=1}
\]
\[
    -27 + \frac{81}{4} = -frac{27}{4}
\]
The answer is $-\frac{27}{4}$
\newpage
\section{Question 10}
Evaulate $\iiint_{\mathcal{W}}f(x, y,z) \, dV$ for the function $f$ and region $\mathcal{W}$ specified
\[
    f(x, y,z) = e^{x+y+z}; \quad \mathcal{W}: 0\leq z \leq 1, \, 0 \leq y \leq x, \, 0 \leq z \leq 1
\]
\[
    \int_0^1 \int_0^1 \int_0^{x} e^{x+y+z} \, dy \, dx \, dz
\]
\[
    \int_0^1 \int_0^1 \left( e^{x+y+z} \Big |_{y=0}^{y=x} \right) \, dx \, dz
\]
\[
    \int_0^1 \int_0^1 e^{2x+z} - e^{x+z} \, dx \, dz
\]
\[
    \int_0^1 \left( \left. \frac{e^{2x+z}}{2}- e^{x+z} \right |_{x = 0}^{x = 1} \right) \, dz
\]
\[
    \int_0^1 \left(\frac{e^{z+2}}{2}- e^{z+1}\right) - \left( \frac{e^z}{2} - e^z\right) \, dz
\]
\[
    \left. \frac{e^{z+2}}{2}- e^{z+1} + \frac{e^z}{2} \right |_{z = 0}^{z = 1}
\]
\[
    \frac{e^3}{2} - e^{2} + \frac{e}{2} - (\frac{e^2}{2} - e + \frac{1}{2}) 
\]
The answer is $\frac{e^3}{2} -\frac{3e^2}{2} + \frac{3e}{2} - \frac{1}{2}$
\newpage
\section{Question 15}
Calculate the integral of $f(x, y,z) =z$ over the region $\mathcal{W}$ in Figure 11, below the hemisphere of radius 3 and lying over the triangle $D$ in the $xy-$plane bounded by $x=1,\, y=1, \, x=y$

First, let's calculate the bounds:
\[
    0 \leq x \leq 1 \quad x \leq y \leq 1 \quad 0 \leq z \leq \sqrt{9-x^2-x^2}
\]
\[
    \int_0^1 \int_x^1 \int_0^{\sqrt{9-x^2-y^2}} z \, dz \, dy \, dx
\]
\[
    \int_0^1 \int_x^1 \left( \left. \frac{z^2}{2} \right |_0^{\sqrt{9-x^2-y^2}}\right) \, dy \, dx
\]
\[
    \int_0^1 \int_x^1 \frac{9-x^2-y^2}{2} \, dy \, dx
\]
\[
    \frac{1}{2}\int_0^1 \int_x^1 9-x^2-y^2 \, dy \, dx
\]
\[
    \frac{1}{2}\int_0^1 9y-x^2y-\frac{y^3}{3} \Big |_{x}^{1} \, dx
\]
\[
    \frac{1}{2}\int_0^1 (9-x^2-\frac{1}{3}) - (9x-x^3 - \frac{x^3}{3}) \, dx
\]
\[
    \frac{1}{2} \Big[9x-\frac{x^3}{3} - \frac{x}{3} - \frac{9x^2}{2}+ \frac{x^4}{4} + \frac{x^4}{12} \Big]_{x=0}^{x=1}
\]
\[
    \frac{1}{2} (9 - \frac{1}{3} - \frac{1}{3} - \frac{9}{2} +\frac{1}{4} + \frac{1}{12}) = \frac{25}{12}
\]
The integral evaluates to $\frac{25}{12}$
\newpage
\section{Question 21}
Find the volume of the solid in the first octant bounded between the planes $x+y+z=1$ and $x+y+2z=1$

First, let us find where the two planes intersection
\[
    z = 1 - x - y \quad \quad z = \frac{1-x-y}{2}
\]
\[
    1-x-y = \frac{1-x-y}{2}
\]
\[
    2-2x-2y = 1-x-y
\]
\[
    x+y =1
\]
Let's compute the bounds
\[
0 \leq x \leq 1 \quad \quad 0 \leq x \leq 1-x 
\]
\[
    \int_{0}^1 \int_{0}^{1-x} 1-x-y -\frac{1-x-y}{2} \, dx
\]
\[
    \int_{0}^1 \int_{0}^{1-x} 1-x-y -\frac{1-x-y}{2} \, dx
\]
\[
    \frac{1}{2}\int_{0}^1 \int_{0}^{1-x}1-x-y \, dx
\]
\[
    \frac{1}{2}\int_{0}^1\left( y-xy-\frac{y^2}{2} \Big |_{y=0}^{y =1-x}\right)\, dx
\]
\[
    (1-x) - (1-x)x - \frac{(1-x)^2}{2}
\]
\[
    1-2x+x^2 -  \frac{(1-2+x^2}{2} = \frac{1-2x+x^2}{2} 
\]
\[
    \frac{1}{4} \int_0^1 1-2x+x^2
\]
\[
    \frac{1}{4} \Big [ x - x^2 + \frac{x^3}{3} -\Big]_{x=0}^{x=1}
\]
\[
    \frac{1}{4} (1 - 1 + \frac{1}{3})
\]
\[
    \frac{1}{4} \cdot \frac{1}{3}
\]
The answer is $\frac{1}{12}$
\newpage
\section{Question 22}
Evalulate $\iiint_{\mathcal{W}} y \, dV$ where $\mathcal{W}$ is the region above $z = x^2 + y^2$ and below $z = 5$, and bounded by $y = 0$ and $y = 1$. 
First, let us define the bounds:
\[
    x^2 + y^2 \leq z \leq 5
\]
\[
    0 \leq y \leq 1
\]
To solve for the bounds of x, solve for the intersection of the two surfaces:
\[
    x^2 + y^2 = 5
\]
\[
    x^2 = 5 - y^2
\]
\[
    x = \pm \sqrt{5 - y^2}
\]
\[
    -\sqrt{5-y^2} \leq x \leq \sqrt{5-y^2}
\]
Now, we can rewrite the integral as:
\[
    \int_{0}^{1} \int_{-\sqrt{5-y^2}}^{\sqrt{5-y^2}} \int_{x^2 + y^2}^{5} y \, dz \, dx \, dy
\]
\[ 
    \int_{0}^{1} \int_{-\sqrt{5-y^2}}^{\sqrt{5-y^2}} \Big ( yz \Big |_{z = x^2 + y^2}^{z = 5} \Big ) dx \, dy
\]
\[ 
    \int_{0}^{1} \int_{-\sqrt{5-y^2}}^{\sqrt{5-y^2}} 5y - y(x^2 + y^2) dx \, dy
\]
\[ 
    \int_{0}^{1} \int_{-\sqrt{5-y^2}}^{\sqrt{5-y^2}} y(5 - x^2 - y^2) dx \, dy
\]
\[
    \int_{0}^{1} \frac{y(15x - x^3 - 3y^2x)}{3} \Big |_{x = -\sqrt{5-y^2}}^{x = \sqrt{5-y^2}} \Big) dy
\]
\[
    \int_{0}^{1} \frac{yx(15 - x^2 - 3y^2)}{3} \Big |_{x = -\sqrt{5-y^2}}^{x = \sqrt{5-y^2}} \Big) dy
\]
\[
    \frac{(y(\sqrt{5-y^2}))(15-(5-y^2)-3y^2)}{3} -
    \frac{(y(-\sqrt{5-y^2}))(15-(5-y^2)-3y^2)}{3}  
\]
\[
    2(\frac{2y\sqrt{5-y^2}(20 - 4y^2)}{3})
\]
\[
    \frac{4(y\sqrt{5-y^2}(5 -y^2))}{3}
\]
\[
    \frac{4y(5-y^2)^\frac{3}{2}}{3}
\]
\[
    \int_{0}^{1}\frac{4y(5-y^2)^\frac{3}{2}}{3}\, dy
\]
\[
    \text{Let $u = 5 - y^2$} \quad du = -2y \,dy \quad -2\,du = 4y\,dy 
\]
\[
    \int_{5}^4 \frac{-2u^\frac{3}{2}}{3} \, du 
\]
\[
    \frac{2}{3} \left( \left . \frac{2u^{\frac{5}{2}}}{5}\right |_4^5\right)
\]
\[
    \frac{4}{5} (5^{\frac{5}{2}} - 4^{\frac{5}{2}})
\]

\end{document}

