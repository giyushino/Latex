\documentclass[letterpaper,12pt]{article}
\def\theme{MidnightBlue}
\usepackage[style=ieee,url=false,doi=false,maxbibnames=99,sorting=ydnt,dashed=false]{biblatex}
\usepackage{geometry}
\usepackage{color}
\usepackage{fontawesome}
\usepackage{parskip}
\usepackage{dashrule}
\usepackage{fancyhdr}
% \usepackage[hidelinks]{hyperref}
\usepackage{lastpage}
\usepackage[
    colorlinks=true,         % Use colored text for links instead of boxes
    linkcolor=,          % Color for internal links
    urlcolor=blue,           % Color for URLs
    citecolor=blue,          % Color for citations
    pdfborder={0 0 0}        % Remove border around links
]{hyperref}

\definecolor{red}{RGB}{218, 0, 55}
% \definecolor{aqua}{RGB}{0, 109, 180u
\definecolor{aqua}{RGB}{57, 141, 137}


\usepackage[dvipsnames]{xcolor}
\usepackage[T1]{fontenc}
\usepackage{lmodern}

\newcommand{\code}[1]{\fontfamily{lmss}\selectfont #1}
\newcommand*\tick{\item[\faAngleRight]}
\renewcommand{\headrulewidth}{0pt}
\renewcommand{\footrulewidth}{0pt}

\geometry{top=1.4cm, bottom=1.4cm, left=1cm, right=1cm}

% \bibliography{papers}

\begin{document}
\pagestyle{fancy}
\fancyhead{}
\fancyfoot{}
\fancyfoot[LO,CE]{\code{\textcolor{aqua}{Last Updated \today}}}
% \fancyfoot[CO,RE]{\code{\thepage}/\textcolor{aqua}3}
\fancyfoot[CO,RE]{\code{\thepage/\textcolor{aqua}{\pageref{LastPage}}}}

\begin{center}
    \Huge {\fontfamily{lmss}\selectfont
        \textbf{Pavan K. \textcolor{aqua}{Yeddanapudi
}}}\\
\normalsize \color{aqua} \code{Undergraduate Student $\bullet$ Researcher}
\end{center}

\begin{center}
    \color{aqua}{\normalsize\faEnvelope}{ \code {\href{mailto:pky@mit.edu}{\color{aqua}pky@mit.edu}}} $|$ {\large\faMobilePhone}{ \code{\href{tel:2092195299}{\color{aqua}+1 209-219-5299 }}} $|$ {\normalsize\faGithub}{ \code {\href{https://github.com/pyfb78}{\color{aqua}pyfb78}}} %$|$ {\normalsize\faHome} {\code {anisha-yeddanapudi.github.io}}
\end{center}
\hfill \\
\begin{minipage}{18cm}
    \Large\code{\textcolor{aqua}{\textbf{Education}}} \textcolor{aqua}{\rule[0.5ex]{16.25cm}{1px}} \hfill  
    \begin{minipage}{19cm}
\hfill\\
    \large \code{\textbf{Massachusetts Institute of Technology}} \hfill \large{Sept 2024 - Today} \\
    \normalsize \code{B.S in Electrical Engineering and Computing} \hfill \normalsize Cambridge, MA\\
    % \small \code{ \textit{Thesis Title} } \\

    %%%%%%%%%%%%%%%%%%
    % \large \code{\textbf{University of California, Berkeley}} \hfill \large{Aug 2021 - May 2024} \\
    % \normalsize \code{BA in Physics \& Applied Math --- Honors in Physics} \hfill \normalsize Berkeley, CA\\
    % \small \code{ GPA: 3.81 }\\

    %%%%%%%%%%%%%%%
    \large \code{\textbf{Las Positas College}} \hfill \large{Jan 2024 - May 2024} \\
    \normalsize \code{Concurrent enrollment during HS} \hfill \normalsize Livermore, CA\\
    \small \code{ GPA: 4.0 }\hfill \\
\end{minipage}
\end{minipage}

\begin{minipage}{18cm}
    \Large\code{\textcolor{aqua}{\textbf{Experience}}} \textcolor{aqua}{\rule[0.5ex]{16.25cm}{1px}} \\
% \begin{minipage}{19cm}
    \hfill\\ \large\color{black}\code{\textbf{Signals, Information, and Algorithms Laboratory}} \hfill Sept 2024 - Today \\
    \normalsize \code{} \hfill \normalsize Cambridge, MA \\
     \small Advisor(s): Dr. Jongha Ryu \hfill \normalsize \color{white} Cambridge, MA 
    
    \color{darkgray}\begin{itemize}
        \tick Read research papers on using Neural Networks to approximate Mutual information between two \newline
        probability distributions
        \tick Implementing algorithm involving utilizing NeuralSVD framework to find eigenfunctions for the kernal function and passing them through KSG estimator to then find approximation for mutual information
        \tick Reading and implementing algorithms involving using representation theory to minimize dimension of data in order to speed up calculations and approximate important processes for analyzing data
    \end{itemize}

    \textcolor{aqua}{\rule[0.5ex]{19.25cm}{1px}}
    
    % \hfill\\ \large\color{black}\code{\textbf{MIT Media Lab}} \hfill Sept 2024 - Today \\
    % \normalsize \code{} \hfill \normalsize Cambridge, MA \\
    %  \small Advisor(s): Mr. Abhishek Singh \hfill \normalsize \color{white} Cambridge, MA 
     \large\color{black} \code{\textbf{MIT Media Lab}} \hfill \large{Sept 2024 - Nov 2024} \\
    \normalsize \code{} \hfill \normalsize Cambridge, MA \\
        \small Advisor(s): Mr. Abhishek Singh \hfill \normalsize \color{white} Cambridge, MA 

    
    \color{darkgray}\begin{itemize}
        \tick Writing code to optimize program for training machine learning models in a decentralized manner
        \tick Devloping algorithms that ensure models are collaboratively trained across a permission-less network, balancing the need for personalization with collective training
        \tick Implemented testing infrastructure in order to allow for continuous development and integration in an open-source environment
        \tick Conducting experiments in order to evaluate performance of decentralized AI models under different conditions, minimizing communication overhead, privacy leakage, and optimizing model accuracy
\end{itemize}

% \end{minipage} 
    \hfill\\ \large\color{black}\code{\textbf{San Jose BioRob Lab}} \hfill August 2023 - May 2024 \\
    \normalsize \code{} \hfill \normalsize San Jose, CA \\
     \small Advisor(s): Prof. Lin Jiang (PI), Mr. Tadeas Horn \hfill \normalsize \color{white} San Jose, CA 

    % \large\color{black} \code{\textbf{San Jose BioRob Lab}} \hfill \large{August 2023 - May 2024} \\
    % \normalsize \code{} \hfill \normalsize San Jose, CA \\
    %     \small Advisor(s): Prof. Lin Jiang (PI), Mr. Tadeas Horn \hfill \normalsize \color{white} San Jose, CA
        \color{darkgray}\begin{itemize}
            \tick Created embedded circuitry in order to make electronics for touch-sensitive glove that lets the user receive feedback for tasks they are currently doing, such as a doctor getting an assessment during surgery
            \tick Implemented matlab simulations to map out movement or robotic glove in real time and collect data to analyze and give feedback to the doctor
            \tick Produced unity simulation for seeing what the position of the glove is in real-time on the screen, allowing for future innovations such as simulations for doctors
    \end{itemize} 
\end{minipage}

\newpage

\begin{minipage}{18cm}
    % \Large\code{\textcolor{aqua}{\textbf{Experience}}} \textcolor{aqua}{\rule[0.5ex]{16.25cm}{1px}} \\
% \begin{minipage}{19cm}
    \hfill\\ \large\color{black}\code{\textbf{UCSB Math Department}} \hfill August 2023 - May 2024 \\
    % \normalsize \code{} \hfill \normalsize Cambridge, MA \\
     \small Advisor(s): Dr. Daryl Cooper \hfill \normalsize \color{white} Cambridge, MA 
    
    \color{darkgray}\begin{itemize}
        \tick Explored relation between the surface area and volume of a multidimensional polygons and some of it's other properties 
        \tick Proved and reinterpreted existing theorem between the volume of multidimensional sphere and Euler's Identity
        \tick Used interpretation to explore the field of linear programming and optimization problems that involve higher dimensional polygons and shapes
\end{itemize}
    
\end{minipage}

\begin{minipage}{18cm}
    \Large\code{\textcolor{aqua}{\textbf{Personal Projects}}} \textcolor{aqua}{\rule[0.5ex]{16.25cm}{1px}} \\
% \begin{minipage}{19cm}
    \hfill\\ \large\color{black}\code{\textbf{Ball Catching Robotic Arm}} \hfill April 2022 - Feb 2024 \\
    % \normalsize \code{} \hfill \normalsize Cambridge, MA \\
     \small  \hfill \normalsize \color{white} Cambridge, MA 
    
    \color{darkgray}\begin{itemize}
        \tick Constructed \href{https://github.com/pyfb78/Robotic-Arm-Ball-Tracker}{\textcolor{blue}{stepper motor library}} from scratch in Cpp, allowing the arm to have seamless movement in whatever direction I want
        \tick Programmed the machine to track a ball using live camera detection with communication utilizing a serial port connection from the python end to the cpp end
        \tick Read \href{http://wpage.unina.it/fabio.ruggiero/Papers/J5.pdf}{\color{blue}research papers} to understand how to camera-in-hand system for the arm to be able to attempt the catch the ball in real time
        \tick Used linear interpolation in order to create \href{https://github.com/pyfb78/Robotic-Arm-Ball-Tracker/blob/main/PythonBallTracking/old_files/ball_trajectory.py}{\color{blue}system} that can track the trajectory of a moving ball and will allow the robot to analyze where it will end up and how to reach a future point along the path
        \tick Utilized the tensorflow library to create a \href{https://github.com/pyfb78/Robotic-Arm-Hand-Gesture-Control}{\color{blue}program} that allows me to control the robotic arm itself using gesture recognition
    \end{itemize}
    
   \hfill\\ \large\color{black}\code{\textbf{Neural Network for Image Recognition}} \hfill Nov 2023 - Jan 2024 \\
    % \normalsize \code{} \hfill \normalsize Cambridge, MA \\
     \small  \hfill \normalsize \color{white} Cambridge, MA 
    
    \color{darkgray}\begin{itemize}
        \tick Implemented \href{https://github.com/pyfb78/Image-Recognition/}{\color{blue} VGG-19 Neural Network} in the tensorflow library in order to analyze and identify images in the open-source CIFAR-100 dataset
        \tick Read and analyzed \href{https://arxiv.org/pdf/1409.1556}{\color{blue}research paper} on how to implement said network, constructing the neural network from the information I had read
        \tick Created and modified my own implementation of the \href{https://github.com/pyfb78/Image-Recognition/blob/main/train.py}{\color{blue}{Cross Entropy Loss}} function in order to better suit the dataset that was being analyzed and understood
    \end{itemize}

\end{minipage}

\end{document}
