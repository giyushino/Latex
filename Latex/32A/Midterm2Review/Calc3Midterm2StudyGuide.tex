\documentclass[hidelinks]{article}
\usepackage[a4paper, total={7in, 10in}]{geometry}
\usepackage[dvipsnames]{xcolor}
\usepackage{amsmath}
\usepackage{tikz}
\usepackage{tkz-euclide}
\usepackage[unicode]{hyperref}
\usepackage[all]{hypcap}
\usepackage{fancyhdr}
\usepackage{amsfonts}

\usetikzlibrary{angles,calc, decorations.pathreplacing}

\definecolor{carminered}{rgb}{1.0, 0.0, 0.22}
\definecolor{capri}{rgb}{0.0, 0.75, 1.0}
\definecolor{brightlavender}{rgb}{0.75, 0.58, 0.89}
\title{\textbf{MATH 32A Midterm 2 Study Guide}}
\author{Allan Zhang the Fuckin Goat}
\date{November 20th, 2024}
\begin{document}
\hypersetup{bookmarksnumbered=true,}
\definecolor{darkspringgreen}{rgb}{0.09, 0.45, 0.27}
\definecolor{darkseagreen}{rgb}{0.56, 0.74, 0.56}
\definecolor{green(munsell)}{rgb}{0.0, 0.66, 0.47}
\pagecolor{white}
\color{black}
\maketitle

\section{14.5 Motion in 3-Space}


the velocity is $\textbf{v}(t) = \textbf{r}'(t)$ and the acceleration is $\textbf{a}(t) = \textbf{v}'(t)$ \\
Likewise, we can also easily extrapolate that $\int{\textbf{a}(t) dt} = \textbf{v}(t)$ and $\int\textbf{v}(t) dt = \textbf{r}(t)$ \\\\
Additionally, we can decompose $\textbf{a}(t)$ into the sum of tangential and normal components. Remember that 


\[
	\textbf{T}(t) = \frac{\textbf{v}(t)}{\|\textbf{v}(t)\|}, \quad \quad  \textbf{N}(t)= \frac{\textbf{T}(t)}{\|\textbf{T}(t)\|}
\]

Additionally, we know that $\textbf{a} = a_{\textbf{T}}\textbf{T} + a_{\textbf{N}}\textbf{N}$, where the coffeficient $a_{\textbf{T}}$is called the tangential component and $a_{\textbf{N}}$ is the normal component of acceleration. \\
To find the value of the tangential component and the normal component, we can use the following equations: 

\[
	a_{\textbf{T}} = \textbf{a $\cdot$ T} = \frac{\textbf{a $\cdot$ v}}{\|\textbf{v}\|}, \quad \quad a_{\textbf{N}} = \textbf{a} \cdot \textbf{N} = \sqrt{\|\textbf{a}\|^2 - \|a_{\textbf{T}}\|^2}
\] where $\|\textbf{v}\|$ is the magnitude of the velocity vector





$\lim_{(x, y) \to (0, 0)}{\frac{x^2y}{x^4+y^2}}$



test

\end{document}
