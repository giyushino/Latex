\documentclass[hidelinks]{article}
\usepackage[a4paper, total={7in, 10in}]{geometry}
\usepackage[dvipsnames]{xcolor}
\usepackage{amsmath}
\usepackage{tikz}
\usepackage{tkz-euclide}
\usepackage[unicode]{hyperref}
\usepackage[all]{hypcap}
\usepackage{fancyhdr}
\usepackage{amsfonts}

\usetikzlibrary{angles,calc, decorations.pathreplacing}

\definecolor{carminered}{rgb}{1.0, 0.0, 0.22}
\definecolor{capri}{rgb}{0.0, 0.75, 1.0}
\definecolor{brightlavender}{rgb}{0.75, 0.58, 0.89}
\title{\textbf{MATH 32A Problem Set 7}}
\author{Allan Zhang}
\date{November 18th, 2024}
\begin{document}
\hypersetup{bookmarksnumbered=true,}
\definecolor{darkspringgreen}{rgb}{0.09, 0.45, 0.27}
\definecolor{darkseagreen}{rgb}{0.56, 0.74, 0.56}
\definecolor{green(munsell)}{rgb}{0.0, 0.66, 0.47}
\pagecolor{white}
\color{black}
\maketitle


\section{Question 1}
Suppose the tangent plane to a function $g(x, y)$ at a point $P$ has a normal
vector $\langle 0, -1, 1 \rangle$. If you increase the value of $ y$ by a small amount, do you expect that the
function will increase in value, decrease in value, or stay the same? Explain your answer. \\

Let's take into consideration the tangent plane. Let $P = (x_0, y_0)$ The equation of the tangent plane is: \[
0(x-x_0) -1(y-y_0)+(g(x, y) -g(x_0, y_0)) = 0
\]
\[
g(x, y) = y - y_0 + g(x_0, y_0)
\]
From this simplified form, we can see that if we increase the value of $y$, the function $g(x, y)$ will increase as well. 

\newpage
\section{Question 2}
Let $I = W/H^2$ denote body mass index, where $W$ is body weight and $H$ is
the body height. Suppose that $(W, H) = (30, 1.2)$. Use linearization and/or differentials
to estimate the change in height that would cause BMI to change by 3, if weight is held constant. 

\[
	I_{H} = -2\frac{W}{H^3}
\]
\[
        -2\frac{W}{H^3} \cdot \Delta H = 3
\]
\[
	-2\frac{30}{1.2^3} \cdot \Delta H = 3  
\]
\[
	\Delta H = -0.0864
\]

\newpage

\section{Question 3}
Let $g(x, y) = 3\sin^2(x)y^4$. Compute the directional derivative of $g$ in the direction of the vector $\langle 1, 1 \rangle$ at the point $(0, 2)$

\[
	g_x = 6 \sin(x)\cos(x)y^4 \quad \quad 12\sin^2(x)y^4
\]
\[ \nabla g(0, 2) = \langle 6 \sin(x)\cos(x)y^4,  12 \sin^2(x)y^3  \rangle = \langle 
 0,  12 \sin^2(x)y^3  \rangle  = \langle 0, 0 \rangle\]

 \[ D_{\vec{u}}\nabla g = \nabla g \cdot \vec{u} = 
	 \langle 0, 0 \rangle \cdot \frac{1}{\sqrt{2}}\langle 1, 1 \rangle = 0 + 0 = 0 \]

\[
	\text{The directional derivative is simply 0}
\]
\newpage
\section{Question 4}
Let $f(x, y, z) = x^2\sqrt{y}-7z$. Determine the direction from the point $(1, 4, 1)$ in which $f$ is increasing the fastest, and calculate the rate of change in that direction

\[
	\nabla f = \langle 2x\sqrt{y}, \frac{x^2}{2\sqrt{y}}, -7 \rangle
\]
\[
	\nabla f(1, 4, 1) = \langle 4, \frac{1}{4}, -7 \rangle
\]

\[
	\text{rate} = \sqrt{4^2+(\frac{1}{4})^2+(-7)^2} = \sqrt{65 + \frac{1}{16}} = \sqrt{\frac{1041}{16}}
\]

\[
	\text{The direction at which $f$ is increasing the fastest is $ \sqrt{\frac{16}{1041}} \cdot \langle 4, \frac{1}{4}, -7 \rangle
$}
\]
\newpage

\section{Question 5}
Plot level curves for the function $f(x, y) = xy$ at heights -1, 0, and 1. Plot $\nabla(f)$ at the points $(0, 1), and (1, -1)$

hello



\end{document}

