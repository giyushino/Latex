\documentclass[hidelinks]{article}
\usepackage[a4paper, total={7in, 10in}]{geometry}
\usepackage[dvipsnames]{xcolor}
\usepackage{amsmath}
\usepackage{tikz}
\usepackage{tkz-euclide}
\usepackage[unicode]{hyperref}
\usepackage[all]{hypcap}
\usepackage{fancyhdr}
\usepackage{amsfonts}

\usetikzlibrary{angles,calc, decorations.pathreplacing}

\definecolor{carminered}{rgb}{1.0, 0.0, 0.22}
\definecolor{capri}{rgb}{0.0, 0.75, 1.0}
\definecolor{brightlavender}{rgb}{0.75, 0.58, 0.89}
\title{\textbf{MATH 32A Problem Set 7}}
\author{Allan Zhang}
\date{November 18th, 2024}
\begin{document}
\hypersetup{bookmarksnumbered=true,}
\definecolor{darkspringgreen}{rgb}{0.09, 0.45, 0.27}
\definecolor{darkseagreen}{rgb}{0.56, 0.74, 0.56}
\definecolor{green(munsell)}{rgb}{0.0, 0.66, 0.47}
\pagecolor{white}
\color{black}
\maketitle


\section{Question 1}
Suppose the tangent plane to a function $g(x, y)$ at a point $P$ has a normal
vector $\langle 0, -1, 1 \rangle$. If you increase the value of $ y$ by a small amount, do you expect that the
function will increase in value, decrease in value, or stay the same? Explain your answer. \\
Let's take into consideration the tangent plane. Let $P = (x_0, y_0)$ The equation of the tangent plane is: \[
0(x-x_0) -1(y-y_0)+(g(x, y) -g(x_0, y_0)) = 0
\]
\[
g(x, y) = y - y_0 + g(x_0, y_0)
\]
From this simplified form, we can see that if we increase the value of $y$, the function $g(x, y)$ will increase as well. 

\newpage

\end{document}

