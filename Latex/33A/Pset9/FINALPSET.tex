\documentclass[hidelinks]{article}
\usepackage[a4paper, total={7in, 10in}]{geometry}
\usepackage[dvipsnames]{xcolor}
\usepackage{amsmath}
\usepackage{tikz}
\usepackage{tkz-euclide}
\usepackage[unicode]{hyperref}
\usepackage[all]{hypcap}
\usepackage{fancyhdr}
\usepackage{amsfonts}
\usepackage[utf8]{inputenc}
\DeclareUnicodeCharacter{2212}{-}
\usepackage{amsmath}
\mathchardef\mhyphen="2D % Define a "math hyphen"
\newcommand\rnumber{\mathop{r\mhyphen number}}

\usetikzlibrary{angles,calc, decorations.pathreplacing}

\definecolor{carminered}{rgb}{1.0, 0.0, 0.22}
\definecolor{capri}{rgb}{0.0, 0.75, 1.0}
\definecolor{brightlavender}{rgb}{0.75, 0.58, 0.89}
\title{\textbf{MATH 33A Problem Set 9}}
\author{Allan Zhang}
\date{December 8, 2024}
\begin{document}
\hypersetup{bookmarksnumbered=true,}
\definecolor{darkspringgreen}{rgb}{0.09, 0.45, 0.27}
\definecolor{darkseagreen}{rgb}{0.56, 0.74, 0.56}
\definecolor{green(munsell)}{rgb}{0.0, 0.66, 0.47}
\pagecolor{white}
\color{black}
\maketitle

\section{Question 1}
Find an orthonormal eigenbasis for the following matrices:

(a) $\begin{bmatrix} 6 & 2 \\ 2 & 3\end{bmatrix}$

\[
	\text{det}(\begin{bmatrix} 6 & 2 \\ 2 & 3\end{bmatrix} - \lambda I) = \text{det}(\begin{bmatrix} 6 -\lambda & 2 \\ 2 & 3 -\lambda \end{bmatrix})
\]
\[
	(6-\lambda)(3-\lambda) - 4 = 0 
\]
\[
	18 - 9\lambda + \lambda^2 - 4 = 0 \quad \quad 14 -9\lambda + \lambda^2 = 0
\]
\[
	\lambda = 7 \quad \quad \lambda = 2
\]
\[
	\text{Let $\lambda = 2$: ker}(\begin{bmatrix} 6 - 2 & 2 \\ 2 & 3 -2 \end{bmatrix})
\]
\[
	\begin{bmatrix} 4 & 2 & 0 \\ 2 & 1 & 0 \end{bmatrix} =  \begin{bmatrix} 2 & 1 & 0 \\ 0 & 0 & 0 \end{bmatrix}
\]
\[
	2x + y = 0 \quad \quad x = -\frac{1}{2}y \quad \text{The eigenvector is: }\begin{bmatrix} -\frac{1}{2} \\ 1 \end{bmatrix}
\]
\[
	\text{Let $\lambda = 7$: ker}(\begin{bmatrix} 6 - 7 & 2 \\ 2 & 3 -7 \end{bmatrix})
\]
\[
	\begin{bmatrix} -1 & 2 & 0 \\ 2 & -4 & 0 \end{bmatrix} =  \begin{bmatrix} 1 & -2 & 0 \\ 0 & 0 & 0 \end{bmatrix}
\]
\[
	x - 2y = 0 \quad \quad x = 2y \quad \text{The eigenvector is: }\begin{bmatrix} 2 \\ 1 \end{bmatrix}
\]
\[
	\| \begin{bmatrix} -\frac{1}{2} \\ 1 \end{bmatrix}\| = \frac{2}{\sqrt{5}} \begin{bmatrix} -\frac{1}{2} \\ 1 \end{bmatrix} \quad \quad 	\| \begin{bmatrix} 2 \\ 1 \end{bmatrix}\| = \frac{1}{\sqrt{5}} \begin{bmatrix} 2 \\ 1 \end{bmatrix}
\]
\[
	\text{The orthonormal eigenbasis is span} \{ \frac{2}{\sqrt{5}} \begin{bmatrix} -\frac{1}{2} \\ 1 \end{bmatrix}, \frac{1}{\sqrt{5}} \begin{bmatrix} 2 \\ 1 \end{bmatrix}
\}
\]


(b) $\begin{bmatrix}-2 & 1 & 1 \\ 1 & -2 & 1 \\ 1 & 1  & -2 \end{bmatrix}$

\[
	\text{det}(\begin{bmatrix}-2 -\lambda & 1 & 1 \\ 1 & -2 -\lambda & 1 \\ 1 & 1  & -2 -\lambda \end{bmatrix}) = (-2-\lambda)((-2-\lambda)^2 - 1) - 1((-2-\lambda) - 1) + 1(1 - (-2-\lambda))
\]
\[
	(2-\lambda)(\lambda^2-4\lambda+3) -1 (-\lambda- 3) + 1(\lambda + 3) = (2-\lambda)(\lambda^2-4\lambda+3) + \lambda + 3 + \lambda + \lambda^2 + 11\lambda - 6 = 0
\]
\[
	-\lambda^3 -6\lambda^2-11\lambda-6+2\lambda + 6 = 0 \quad \quad \lambda^3 - 6\lambda^2-9\lambda = 0
\]
\[
	\lambda(\lambda -3)^2 = 0 \quad \quad \lambda = 0 \quad \lambda = 3 \text{(multiplicity 2)}
\]
\[
	\begin{bmatrix}-2 & 1 & 1 & 0 \\ 1 & -2 & 1 & 0 \\ 1 & 1  & -2 & 0\end{bmatrix} = \begin{bmatrix}0 & 3 & -3 & 0 \\ 0 & -3 & 3 & 0 \\ 1 & 1  & -2 & 0\end{bmatrix} = \begin{bmatrix}0 & 3 & -3 & 0 \\ 0 & -3 & 3 & 0 \\ 1 & 1  & -2 & 0\end{bmatrix}= \begin{bmatrix}0 & 0 & 0 & 0 \\ 0 & 1 & -1 & 0 \\ 1 & 1  & -2 & 0\end{bmatrix} = \begin{bmatrix}0 & 0 & 0 & 0 \\ 0 & 1 & -1 & 0 \\ 1 & 0  & -1 & 0\end{bmatrix}
\]
\[
	x = z \quad \quad y = z \quad \text{The eigenvector is } \begin{bmatrix} 1 \\ 1 \\ 1 \end{bmatrix}
\]
\[
\begin{bmatrix}1 & 1 & 1 & 0 \\ 1 & 1 & 1 & 0 \\ 1 & 1  & 1 & 0\end{bmatrix} = \begin{bmatrix} 1 & 1 & 1 & 0 \\ 0 & 0 & 0 & 0 \\ 0 & 0  & 0 & 0\end{bmatrix} 
\]
\[
x = -y - z
\]
\[
\text{The span is: } \{ \begin{bmatrix} -1 \\ 1 \\0 \end{bmatrix}, \begin{bmatrix} -1 \\ 0 \\ -1 \end{bmatrix}\}
\]
\[
	u_1 = \frac{1}{\sqrt{2}}\begin{bmatrix} -1 \\ 1 \\0 \end{bmatrix}
\]
\[
	u_2 =  \frac{v_2 - \text{proj}_{u_1} v_2}{\| v_2 - \text{proj}_{u_1} v_2\|} = \frac{v_2 - (u_1 \cdot v_2)u_1}{\|v_2 - (u_1 \cdot v_2)u_1\|} = \frac{\begin{bmatrix} -1 \\ 0 \\ -1 \end{bmatrix} - 1\begin{bmatrix} -1 \\ 1 \\ 0 \end{bmatrix}}{\sqrt{2}}= \frac{\begin{bmatrix} 0 \\ -1 \\ -1 \end{bmatrix}}{\sqrt{2}}
\]
\[
	\text{The orthonormal eigenbasis is } \{ \frac{1}{\sqrt{3}} \begin{bmatrix} 1 \\ 1 \\ 1 \end{bmatrix} ,\frac{1}{\sqrt{2}} \begin{bmatrix} -1 \\ 1 \\ 0 \end{bmatrix},\frac{1}{\sqrt{2}} \begin{bmatrix} 0 \\ -1 \\ -1 \end{bmatrix} \}
\]

\newpage
\section{Question 2}
Find a 2 x 2 matrix $A$ such that $A^3 = \begin{bmatrix} 13 & 14 \\ 14 & 13 \end{bmatrix}$

\[
	P^{-1}AP = D
\]
\[
	\text{det}(\begin{bmatrix} 13 & 14 \\ 14 & 13 \end{bmatrix} - \lambda I) = \text{det}(\begin{bmatrix} 13 - \lambda & 14 \\ 14 & 13 - \lambda \end{bmatrix}) = (13 - \lambda)^2 - 14^2 = 0  
\]
\[
	169 - 26\lambda + \lambda^2 - 196 = \lambda^2 - 26\lambda - 27 = (\lambda - 27)(\lambda + 1)
\]
\[
	\lambda = 27 \quad \quad \lambda = -1
\]
\[
	\text{ker}(\begin{bmatrix} 13 & 14 \\ 14 & 13 \end{bmatrix} - 27 I) = \text{ker}(\begin{bmatrix} -14 & 14 \\ 14 & -14 \end{bmatrix})
= \text{ker}(\begin{bmatrix} 1 & -1 \\ 0 & 0 \end{bmatrix}) 
\]
\[
	x = y \quad \text{The corresponding eigenvector is }\begin{bmatrix} 1 \\ 1 \end{bmatrix}
\]

\[
	\text{ker}(\begin{bmatrix} 13 & 14 \\ 14 & 13 \end{bmatrix} - (-1) I) = \text{ker}(\begin{bmatrix} 14 & 14 \\ 14 & 14 \end{bmatrix})
= \text{ker}(\begin{bmatrix} 1 & 1 \\ 0 & 0 \end{bmatrix})
\]
\[
	x = -y \quad \text{The corresponding eigenvector is }\begin{bmatrix} -1 \\ 1 \end{bmatrix}
\]
\[
	P = \begin{bmatrix} 1 & -1 \\ 1 & 1 \end{bmatrix} 
\]
\[
	\begin{bmatrix} 1 & -1 & 1 & 0 \\ 1 & 1 & 0 & 1 \end{bmatrix} = 
	\begin{bmatrix} 1 & 1 & 0 & 1 \\ 1 & -1 & 1 & 0 \end{bmatrix} = 
	\begin{bmatrix} 1 & 1 & 0 & 1 \\ 0 & -2 & 1 & -1 \end{bmatrix} = 
	\begin{bmatrix} 1 & 1 & 0 & 1 \\ 0 & 1 & -1/2 & 1/2 \end{bmatrix} = 
	\begin{bmatrix} 1 & 0 & 1/2 & 1/2 \\ 0 & 1 & -1/2 & 1/2 \end{bmatrix}
\]
\[
	P^{-1} = 
	\begin{bmatrix} 1/2 & 1/2 \\ -1/2 & 1/2 \end{bmatrix}
\]
\[
	\begin{bmatrix} 1/2 & 1/2 \\ -1/2 & 1/2 \end{bmatrix} 	
	\begin{bmatrix} 13 & 14 \\ 14 & 13 \end{bmatrix}	
	\begin{bmatrix} 1 & -1 \\ 1 & 1 \end{bmatrix}= D			
\]
\[	
	\begin{bmatrix} 1/2 & 1/2 \\ -1/2 & 1/2 \end{bmatrix} 	
	\begin{bmatrix} 27 & 1 \\ 27 & -1 \end{bmatrix}	= D
\]
\[	
	\begin{bmatrix} 27 & 0 \\ 0 & -1 \end{bmatrix} 	
\]
\[
	A = P{-1}D^{-1/3}P = 
	\begin{bmatrix} 1/2 & 1/2 \\ -1/2 & 1/2 \end{bmatrix} 		
	\begin{bmatrix} 3 & 0 \\ 0 & -1 \end{bmatrix} 
	\begin{bmatrix} 1 & -1 \\ 1 & 1 \end{bmatrix} 	
\]
\[\begin{bmatrix} 1/2 & 1/2 \\ -1/2 & 1/2 \end{bmatrix} 		
\begin{bmatrix} 3 & -3 \\ -1 & -1 \end{bmatrix} =  
\begin{bmatrix} 1 & 2 \\ 2 & 1 \end{bmatrix}
\]
\newpage
\section{Question 3}
Find the symmetric matrix of the following quadratic form. Also, determine whether it is positive definite, negative definite, or indefinite.
\[
	\text{(a)} \quad q(x_1, x_2) = x_1^2 + 4x_1x_2 + 4x_2^2
\]
\[
	\begin{bmatrix} 1 & -2 \\ -2 & 4 \end{bmatrix} 
\]
\[
	\text{det}(\begin{bmatrix} 1 - \lambda & -2 \\ -2 & 4 - \lambda \end{bmatrix}) = (1-\lambda)(4-\lambda) - 4 = 0
\]
\[
	4 - 5\lambda + \lambda^2 - 4 = \lambda^2 - 5\lambda = \lambda(\lambda - 5) = 0
\]
\[
	\lambda = 0 \quad \quad \lambda = 5
\]
Since one of the eigenvalues is positive and the other is negative, the quadratic form is positive semidefinite

\[
	\text{(b)} \quad q(x_1, x_2) = 2x_1^2 + 11x_1x_2 + 9x_2^2
\]
\[
	\begin{bmatrix} 2 & 11/2 \\ 11/2 & 9 \end{bmatrix}
\]
\[
	\text{det}( \begin{bmatrix} 2 - \lambda & 11/2 \\ 11/2 & 9 - \lambda \end{bmatrix}) = (2-\lambda)(9-\lambda) - 121/4 = 0
\]
\[
	18 - 11\lambda +  \lambda^2 - 121/4 = 72/4 -121/4 + \lambda^2 - 11\lambda = \lambda^2 - 11\lambda - 49/4 = 0 
\]
\[
	\lambda = \frac{11 \pm \sqrt{121 + 49}}{2}
\]
From this, we know that one of the eigenvalues is positive and the other is negative. We can conclude that this quadratic form is indefinite.

\[
	\text{(c)} \quad q(x_1, x_2, x_3) = 2x_1^2 + 2x_2^2 + 2x_3^2 -2x_1x_2-2x_1x_2
\]
\[
	\begin{bmatrix} 2 & -1 & -1 \\ -1 & 2 & 0 \\ -1 & 0 & 2 \end{bmatrix}
\]
\[
	\text{det}(\begin{bmatrix} 2 -\lambda & -1 & -1 \\ -1 & 2 -\lambda & 0 \\ -1 & 0 & 2 -\lambda \end{bmatrix})
\]
\[
	(2-\lambda)(-1 + (2-\lambda)) -1(2-\lambda) -1(2-\lambda)^2 = 0
\]
\[
	(2-\lambda)(\lambda^2-4\lambda + 2) = 0
\]
\[
	\lambda = 2 \quad \quad \lambda = -\sqrt{2} + 2\quad \quad \lambda = -\sqrt{2} + 2
\]
Since all the eigenvalues are positive, the quadratic form is positive definite.

\newpage
\section{Question 5}
A = $\begin{bmatrix} 2 & 3 \\ 0 & 2 \end{bmatrix}$. Find unit vectors $\vec{u}_1$ and $\vec{u_2}$ such that $\| A\vec{u}_1\| = 4$ and $\| A\vec{u}_2\| = 1$
\[
	A^T = \begin{bmatrix} 2 & 0 \\ 3 & 2 \end{bmatrix}
\]
\[
	A^TA = \begin{bmatrix} 2 & 0 \\ 3 & 2 \end{bmatrix} \begin{bmatrix} 2 & 3 \\ 0 & 2 \end{bmatrix} = \begin{bmatrix} 4 & 6 \\ 6 & 13 \end{bmatrix} 
\]
\[
	\text{det}(\begin{bmatrix} 4 - \lambda & 6 \\ 6 & 13 - \lambda \end{bmatrix}) = (4-\lambda)(13-\lambda) - 36 = 0
\]
\[
	52 - 17\lambda + \lambda^2 - 36 = \lambda^2 = 17\lambda + 16 = (\lambda - 1)(\lambda - 16) = 0
\]
\[
	\lambda = 1 \quad \quad \lambda = 16
\]
\[
	\text{ker}(\begin{bmatrix} 4 - 1 & 6 \\ 6 & 13 - 1 \end{bmatrix}) 
\]
\[
	\begin{bmatrix} 3 & 6 & 0 \\ 6 & 12 & 0 \end{bmatrix} = \begin{bmatrix} 1 & 2 & 0 \\ 0 & 0 & 0 \end{bmatrix} 
\]
\[
	x = -2y \quad \quad \text{ker}(\begin{bmatrix} 3 & 6 \\ 6 & 12 \end{bmatrix}) = \{\begin{bmatrix} -2y \\ y \end{bmatrix} : y \in \mathbb{R}\} 
\]
\[
	\| \begin{bmatrix} -2 \\ 1 \end{bmatrix} \| = \sqrt{5} 
\[
	\vec{u_1} = \sqrt{5} \begin{bmatrix} -2 \\ 1 \end{bmatrix}
\]
\[
	\text{ker}(\begin{bmatrix} 4 - 16 & 6 \\ 6 & 13 - 16 \end{bmatrix}) 
\]
\[
	\begin{bmatrix} -12 & 6 & 0 \\ 6 & -3 & 0 \end{bmatrix} = \begin{bmatrix} 2 & -1 & 0 \\ 0 & 0 & 0 \end{bmatrix} 
\]
\[
	x = \frac{1}{2}y \quad \quad \text{ker}(\begin{bmatrix} -12 & 6 \\ 6 & -3 \end{bmatrix}) = \{\begin{bmatrix} \frac{y}{2} \\ y \end{bmatrix} : y \in \mathbb{R}\} 
\]
\[
	\| \begin{bmatrix} \frac{1}{2} \\ 1 \end{bmatrix} \| = \frac{\sqrt{5}}{2} 
\]
\[
	\vec{u_2} = \frac{2}{\sqrt{5}} \begin{bmatrix} \frac{1}{2} \\ 1 \end{bmatrix} = \frac{1}{\sqrt{5}} \begin{bmatrix} 1 \\ 2 \end{bmatrix}
\]

\end{document}
