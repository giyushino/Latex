\documentclass[10pt, letterpaper]{article}

% Packages:
\usepackage[
    ignoreheadfoot, % set margins without considering header and footer
    top=2 cm, % seperation between body and page edge from the top
    bottom=2 cm, % seperation between body and page edge from the bottom
    left=2 cm, % seperation between body and page edge from the left
    right=2 cm, % seperation between body and page edge from the right
    footskip=1.0 cm, % seperation between body and footer
    % showframe % for debugging 
]{geometry} % for adjusting page geometry
\usepackage{titlesec} % for customizing section titles
\usepackage{tabularx} % for making tables with fixed width columns
\usepackage{array} % tabularx requires this
\usepackage[dvipsnames]{xcolor} % for coloring text
\definecolor{primaryColor}{RGB}{0, 79, 144} % define primary color
\usepackage{enumitem} % for customizing lists
\usepackage{fontawesome5} % for using icons
\usepackage{amsmath} % for math
\usepackage[
    pdftitle={Allan Zhang's CV},
    pdfauthor={Allan Zhang},
    pdfcreator={LaTeX with RenderCV},
    colorlinks=true,
    urlcolor=primaryColor
]{hyperref} % for links, metadata and bookmarks
\usepackage[pscoord]{eso-pic} % for floating text on the page
\usepackage{calc} % for calculating lengths
\usepackage{bookmark} % for bookmarks
\usepackage{lastpage} % for getting the total number of pages
\usepackage{changepage} % for one column entries (adjustwidth environment)
\usepackage{paracol} % for two and three column entries
\usepackage{ifthen} % for conditional statements
\usepackage{needspace} % for avoiding page brake right after the section title
\usepackage{iftex} % check if engine is pdflatex, xetex or luatex

% Ensure that generate pdf is machine readable/ATS parsable:
\ifPDFTeX
    \input{glyphtounicode}
    \pdfgentounicode=1
    % \usepackage[T1]{fontenc} % this breaks sb2nov
    \usepackage[utf8]{inputenc}
    \usepackage{lmodern}
\fi



% Some settings:
\AtBeginEnvironment{adjustwidth}{\partopsep0pt} % remove space before adjustwidth environment
\pagestyle{empty} % no header or footer
\setcounter{secnumdepth}{0} % no section numbering
\setlength{\parindent}{0pt} % no indentation
\setlength{\topskip}{0pt} % no top skip
\setlength{\columnsep}{0cm} % set column seperation
\makeatletter
\let\ps@customFooterStyle\ps@plain % Copy the plain style to customFooterStyle
\patchcmd{\ps@customFooterStyle}{\thepage}{
    \color{gray}\textit{\small Allan Zhang - Page \thepage{} of \pageref*{LastPage}}
}{}{} % replace number by desired string
\makeatother
\pagestyle{customFooterStyle}

\titleformat{\section}{\needspace{4\baselineskip}\bfseries\large}{}{0pt}{}[\vspace{1pt}\titlerule]

\titlespacing{\section}{
    % left space:
    -1pt
}{
    % top space:
    0.3 cm
}{
    % bottom space:
    0.2 cm
} % section title spacing

\renewcommand\labelitemi{$\circ$} % custom bullet points
\newenvironment{highlights}{
    \begin{itemize}[
        topsep=0.10 cm,
        parsep=0.10 cm,
        partopsep=0pt,
        itemsep=0pt,
        leftmargin=0.4 cm + 10pt
    ]
}{
    \end{itemize}
} % new environment for highlights

\newenvironment{highlightsforbulletentries}{
    \begin{itemize}[
        topsep=0.10 cm,
        parsep=0.10 cm,
        partopsep=0pt,
        itemsep=0pt,
        leftmargin=10pt
    ]
}{
    \end{itemize}
} % new environment for highlights for bullet entries


\newenvironment{onecolentry}{
    \begin{adjustwidth}{
        0.2 cm + 0.00001 cm
    }{
        0.2 cm + 0.00001 cm
    }
}{
    \end{adjustwidth}
} % new environment for one column entries

\newenvironment{twocolentry}[2][]{
    \onecolentry
    \def\secondColumn{#2}
    \setcolumnwidth{\fill, 4.5 cm}
    \begin{paracol}{2}
}{
    \switchcolumn \raggedleft \secondColumn
    \end{paracol}
    \endonecolentry
} % new environment for two column entries

\newenvironment{header}{
    \setlength{\topsep}{0pt}\par\kern\topsep\centering\linespread{1.5}
}{
    \par\kern\topsep
} % new environment for the header

\newcommand{\placelastupdatedtext}{% \placetextbox{<horizontal pos>}{<vertical pos>}{<stuff>}
  \AddToShipoutPictureFG*{% Add <stuff> to current page foreground
    \put(
        \LenToUnit{\paperwidth-2 cm-0.2 cm+0.05cm},
        \LenToUnit{\paperheight-1.0 cm}
    ){\vtop{{\null}\makebox[0pt][c]{
        \small\color{gray}\textit{Last updated in December 2024}\hspace{\widthof{Last updated in December 2024}}
    }}}%
  }%
}%

% save the original href command in a new command:
\let\hrefWithoutArrow\href

% new command for external links:
\renewcommand{\href}[2]{\hrefWithoutArrow{#1}{\ifthenelse{\equal{#2}{}}{ }{#2 }\raisebox{.15ex}{\footnotesize \faExternalLink*}}}


\begin{document}
    \newcommand{\AND}{\unskip
        \cleaders\copy\ANDbox\hskip\wd\ANDbox
        \ignorespaces
    }
    \newsavebox\ANDbox
    \sbox\ANDbox{}

    \placelastupdatedtext
    \begin{header}
        \textbf{\fontsize{24 pt}{24 pt}\selectfont Allan Zhang}

        \vspace{0.3 cm}

        \normalsize
        \mbox{{\color{black}\footnotesize\faMapMarker*}\hspace*{0.13cm}31 E Funderburg Blvd, Mountain House, CA}%
        \kern 0.25 cm%
        \AND%
        \kern 0.25 cm%
        \mbox{\hrefWithoutArrow{mailto:allanzhang440@gmail.com}{\color{black}{\footnotesize\faEnvelope[regular]}\hspace*{0.13cm}allanzhang440@gmail.com}}%
        \kern 0.25 cm%
        \AND%
        \kern 0.25 cm%
        \mbox{\hrefWithoutArrow{mailto:allanzhang@g.ucla.edu}{\color{black}{\footnotesize\faEnvelope[regular]}\hspace*{0.13cm}allanzhang@g.ucla.edu}}%
        \kern 0.25 cm%
        \AND%
        \kern 0.25 cm%

        \mbox{\hrefWithoutArrow{tel:+90-541-999-99-99}{\color{black}{\footnotesize\faPhone*}\hspace*{0.13cm}609-943-8429}}%
        \kern 0.25 cm%
        \AND%
        \kern 0.25 cm%
        \mbox{\hrefWithoutArrow{https://giyushino.github.io/}{\color{black}{\footnotesize\faLink}\hspace*{0.13cm}https://giyushino.github.io}}%
        \kern 0.25 cm%
        \AND%
        \kern 0.25 cm%        %\mbox{\hrefWithoutArrow{https://www.linkedin.com/in/allan-zhang-928a6431a}{\color{black}{\footnotesize\faLinkedinIn}\hspace*{0.13cm}Allan Zhang}}%
        %\kern 0.25 cm%
        %\AND%
        %\kern 0.25 cm%
        \mbox{\hrefWithoutArrow{https://github.com/giyushino}{\color{black}{\footnotesize\faGithub}\hspace*{0.13cm}giyushino}}%
    \end{header}

    \vspace{0.3 cm - 0.3 cm}

    \section{Education}



        
        \begin{twocolentry}{
            
            
        \textit{Sept 2024 – June 2028}}
            \textbf{University of California, Los Angeles}

            \textit{BS in Applied Mathematics}
        \end{twocolentry}

        \vspace{0.10 cm}
        \begin{onecolentry}
            \begin{highlights}
                \item GPA: 4.00 %(\href{https://example.com}{a link to somewhere})
                \item \textbf{Relevant Coursework:}  Multivariable Calculus, Linear Algebra, Discrete Structures, Computer Science \\ (Math 32A, Math 32B, Math 33A, Math 33B, Math 61A, CS 31)
            \end{highlights}
        \end{onecolentry}

        \vspace{0.2 cm}
        
        \begin{twocolentry}{
            
            
        \textit{July 2024 - Present}}
            \textbf{Self Study}

            \textit{Machine Learning and Data Science}
        \end{twocolentry}

        \vspace{0.10 cm}
        \begin{onecolentry}
            \begin{highlights}
                \item \textbf{Textbooks:}  \textit {Reinforcement Learning:
An Introduction}, \textit {An Introduction to Statistical Learning with Applications in Python}, \textit{Dive into Deep Learning} 
            \item 
            \textbf{Courses:}  Andrew Ng's Machine Learning Specialization

            \end{highlights}
        \end{onecolentry}

    
    \section{Experience}
        \begin{twocolentry}{
        \textit{Los Angeles, CA}    
            
        \textit{Nov 2024 – Present}}
            \textbf{Undergraduate Research Assistant $|$ BigML}
            
            \textit{UCLA}
        \end{twocolentry}

        \vspace{0.10 cm}
        \begin{onecolentry}
            \begin{highlights}
                \item Set up experiments involving vision transformers, 2D projection layer (MLP), and LLMs to study why VLMs perform poorly on spatial reasoning tasks
                \item Used mechanistic interpretability techniques to extract tensors from every layer, analyzed them to understand how each layer processes input data. Created heatmaps using Matplotlib and Seaborn, visualizing model's predictions for every input token
                \item Created custom probing functions to determine if data passed through VLM retained spatial information encoded into the embeddings, specifically before and after the frozen multimodal projector 
                %\item Wrote functions to extract attention heads and visualize the weights each head in every layer assigns to individual input tokens
            \end{highlights}
        \end{onecolentry}
    



    
    \section{Projects}



        
        \begin{twocolentry}{
            
            
        \textit{\href{https://github.com/giyushino/MyOwnDoodleGuesser}{MyOwnDoogleGuesser}}}
            \textbf{Doodle Guesser}
        \end{twocolentry}

        \vspace{0.10 cm}
        \begin{onecolentry}
            \begin{highlights}
                \item Using vision transformer (OpenAI's CLIP-Vit-Large-Patch-14), downstreamed model to predict animal drawings created by users. Collected and processed 1,000,000+ images from 6 classes to create dataset to fine-tune model on. Improved model's accuracy from 54\% to 87\% after fine-tuning
                \item Also created CLIP model from scratch using PyTorch. Wrote custom tokenzier and encoders to embed input labels and images into multi-dimensional vectors. Achieved 70\% accuracy after being trained on subset of previous dataset, only using 200,000 images. %Created functions to split dataset into batches before training/inference, replicated PyTorch's built in softmax and cross entropy loss functions only using NumPy
                \item Both models used to create game similar to Google Doodle where users draw animals from 6 specified classes and model guesses what animal was drawn. Users' drawings converted into 28x28 tensor consisting of 0s and 1s, fed into model for prediction
                
		\item Tools Used: Python, PyTorch, Pygame, Hugging Face, NumPy
            \end{highlights}
        \end{onecolentry}


        \vspace{0.2 cm}

        \begin{twocolentry}{
            
            
        \textit{\href{https://github.com/giyushino/whatToEatAtUCLA}{What2Eat@UCLA}}}
        \textbf{UCLA Dining Assistant}
        \end{twocolentry}

        \vspace{0.10 cm}
        \begin{onecolentry}
            \begin{highlights}
            \item Using Deepseek API and online data, created chatbot to help students decide where to eat at UCLA
            \item Created pipeline to to scrape UCLA's dining hall menus and extract menu items, descriptions, and ingredients for each dining hall. Used BeautifulSoup to access raw HTML files, wrote functions to extract and clean data, and stored it to readable .txt file. 
            \item Provided Deepseek-chat with data about each dining hall, allowing users to ask chatbot for suggestions based on their preferences. Bot can take into account flavor/cuisine preferences, allergens, dietary restrictions, as well as any other criteria for choosing where to eat. 
            \item Currently implementing small, lightweight open source model, encoders, and cosine similarity functions to replicate Deepseek's functionality.
 		    \item Tools Used: Python, bs4, openai, Transformers
            \end{highlights}
        \end{onecolentry}


        \vspace{0.2 cm}
        \begin{twocolentry}{
            
            
        \textit{\href{https://github.com/giyushino/clip-vit-large-patch14-batch}{Fine-Tuning Functions}}}
            \textbf{Efficient Finetuning Pipeline}
        \end{twocolentry}

        \vspace{0.10 cm}
        \begin{onecolentry}
            \begin{highlights}
                \item Created functions to efficiently fine-tune and evaluate models on weak GPUs or CPU. Functions focused on batching images fed into model to reduce VRAM requirements and preventing Google Colab from crashing
                \item Tested on OpenAI's CLIP-Vit-Large-Patch-14 model using CIFAR-10 dataset. Using built-in training functions from Hugging Face caused Google Colab to crash, custom functions did not. Saw 5\% improvement in accuracy (91\% $\rightarrow$ 96\%) with limited training data, computational power, and time. To prevent overfitting, wrote new function to shuffle training dataset for every epoch
                \item Tools Used: Python, PyTorch, Google Colab, Hugging Face
            \end{highlights}
        \end{onecolentry}


     
        \vspace{0.2 cm}
        \begin{twocolentry}{
            
        \textit{Work in Progress}}
            \textbf{Flappy Bird AI}
        \end{twocolentry}

        \vspace{0.10 cm}
        \begin{onecolentry}
            \begin{highlights}
                \item Created own version of Flappy Bird in Python using Pygame. Currently creating neural network that utilizes Q-learning to learn to play the game. Game state is collected at every frame is collected and fed into model, including bird’s vertical position, vertical velocity, and distance from both pipes
                \item Tools used: Python, PyTorch, Pygame
            \end{highlights}
        \end{onecolentry}

        \vspace{0.2 cm}
        \begin{twocolentry}{
            
        \textit{2020 - 2021}}
            \textbf{Self Published Novel}
        \end{twocolentry}

        \vspace{0.10 cm}
        \begin{onecolentry}
            \begin{highlights}
                \item Wrote a 45,000 word novel,  published it online. Amassed 272,000 reads, 7,000 comments, and 6,200 reviews. Reviewed other authors' works,  providing grammatical advice
            \end{highlights}
        \end{onecolentry}
 
    \section{Skills}
        
        \begin{onecolentry}
            \textbf{Programming Languages and Frameworks:} Python, PyTorch, NumPy, Matplotlib, TensorFlow, scikit-learn,  OpenCV, Hugging Face, \LaTeX, C++ (basic)%, HTML (basic), CSS (basic), Bash, Git
        \end{onecolentry}

        \vspace{0.2 cm}

        \begin{onecolentry}
            \textbf{Languages:} English, Korean
        \end{onecolentry}

        \vspace{0.2 cm}


\end{document}


        \begin{twocolentry}{
            
        \textit{\href{https://github.com/giyushino/UCLABathroom}{UCLA Bathroom}}}
            \textbf{Bathroom Usage Prediction}
        \end{twocolentry}

        \vspace{0.10 cm}
        \begin{onecolentry}
            \begin{highlights}
                \item Implemented linear regression model from scratch in Python to predict how many people are using the
                Rieber Hall 7N men’s bathroom based on time and day. Wrote the cost, gradient, and gradient descent
                functions only using NumPy
                \item Tools used: Python, NumPy, Matplotlib
            \end{highlights}
        \end{onecolentry}


   \vspace{0.2 cm}

        \begin{twocolentry}{
            
            
        \textit{\href{https://github.com/giyushino/hand-detection}{Drone Control}}}
            \textbf{Hand-Controlled Drone}
        \end{twocolentry}

        \vspace{0.10 cm}
        \begin{onecolentry}
            \begin{highlights}
                \item Created and labeled dataset consisting of 400+ images of hands pointing up, down, left, right, backwards, and forwards. Trained YOLOv8 model with the data and utilized OpenCV to detect directions
                real time. Fed real-time collected information to FPV drone’s flight computer to control movement
                \item Tools Used: Python, OpenCV, YOLOv8 
            \end{highlights}
        \end{onecolentry}


