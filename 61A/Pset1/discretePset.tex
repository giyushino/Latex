\documentclass[hidelinks]{article}
\usepackage[a4paper, total={7in, 10in}]{geometry}
\usepackage[dvipsnames]{xcolor}
\usepackage{amsmath}
\usepackage{tikz}
\usepackage{tkz-euclide}
\usepackage[unicode]{hyperref}
\usepackage[all]{hypcap}
\usepackage{fancyhdr}
\usepackage{amsfonts}
\usepackage[utf8]{inputenc}
\DeclareUnicodeCharacter{2212}{-}
\usepackage{amsmath}
\usepackage{array}
\mathchardef\mhyphen="2D % Define a "math hyphen"
\newcommand\rnumber{\mathop{r\mhyphen number}}

\usetikzlibrary{angles,calc, decorations.pathreplacing}

\definecolor{carminered}{rgb}{1.0, 0.0, 0.22}
\definecolor{capri}{rgb}{0.0, 0.75, 1.0}
\definecolor{brightlavender}{rgb}{0.75, 0.58, 0.89}
\title{\textbf{MATH 61A Problem Set 1}}
\author{Allan Zhang}
\date{January 11, 2025}
\begin{document}
\hypersetup{bookmarksnumbered=true,}
\definecolor{darkspringgreen}{rgb}{0.09, 0.45, 0.27}
\definecolor{darkseagreen}{rgb}{0.56, 0.74, 0.56}
\definecolor{green(munsell)}{rgb}{0.0, 0.66, 0.47}
\pagecolor{white}
\color{black}
\maketitle
\section*{Problem 1}
Draw the truth table for $P \to R$, and $P \to Q \text{ and } Q \to R$. Explain in words, referrring to the truth tables, why the former statement implies the latter.

\begin{center}
\text{Truth tables for \( P \to R \), \( P \to Q \), and \( Q \to R \)}\\
\vspace{0.2cm}
\begin{tabular}{ccc} % Three columns for the three tables
  % First table: \( P \to R \)
  \begin{tabular}{|c|c|c|}
    \hline
    \( P \) & \( R \) & \( P \to R \) \\
    \hline
    T & T & T \\
    T & F & F \\
    F & T & T \\
    F & F & T \\
    \hline
  \end{tabular}
  &
  % Second table: \( P \to Q \)
  \begin{tabular}{|c|c|c|}
    \hline
    \( P \) & \( Q \) & \( P \to Q \) \\
    \hline
    T & T & T \\
    T & F & F \\
    F & T & T \\
    F & F & T \\
    \hline
  \end{tabular}
  &
  % Third table: \( Q \to R \)
  \begin{tabular}{|c|c|c|}
    \hline
    \( Q \) & \( R \) & \( Q \to R \) \\
    \hline
    T & T & T \\
    T & F & F \\
    F & T & T \\
    F & F & T \\
    \hline
  \end{tabular}
\end{tabular}
\vspace{0.2cm}

\text{Truth table for $P \to Q$ and $Q \to R$}\\
\begin{tabular}{|c|c|c|}
\hline
\( P \to Q \) & \( Q \to R \) & \( P \to Q \text{ and }Q \to R \) \\
\hline
T & T & T \\ 
T & F & F \\
F & T & F \\
F & F & F \\
\hline
\end{tabular}
\end{center}

From these truth tables, we can observe that $P \to R$ is false when $P$ is true and $R$ is false. There are two cases when $P$ is true, and $Q$ is either true or false, providing us with two cases, either $P \to Q$ is true or false. In the case where the statement is true, $Q$ is true, and vice versa. Now, let us consider when $R$ is false. When $Q$ is false but $R$ is true, $Q \to R$ is false. When $Q \text{ and } R$ are false, $Q \to R$ is true.

Using these observations, I will explain how the former statement implies the latter. We can see how  


\end{document}
