\documentclass[hidelinks]{article}
\usepackage[a4paper, total={7in, 10in}]{geometry}
\usepackage[dvipsnames]{xcolor}
\usepackage{amsmath}
\usepackage{tikz}
\usepackage{tkz-euclide}
\usepackage[unicode]{hyperref}
\usepackage[all]{hypcap}
\usepackage{fancyhdr}
\usepackage{amsfonts}
\usepackage[utf8]{inputenc}
\DeclareUnicodeCharacter{2212}{-}
\usepackage{amsmath}
\usepackage{array}
\mathchardef\mhyphen="2D % Define a "math hyphen"
\newcommand\rnumber{\mathop{r\mhyphen number}}

\usetikzlibrary{angles,calc, decorations.pathreplacing}

\definecolor{carminered}{rgb}{1.0, 0.0, 0.22}
\definecolor{capri}{rgb}{0.0, 0.75, 1.0}
\definecolor{brightlavender}{rgb}{0.75, 0.58, 0.89}
\title{\textbf{MATH 61A Problem Set 4}}
\author{Allan Zhang}
\date{February 23, 2025}
\begin{document}
\hypersetup{bookmarksnumbered=true,}
\definecolor{darkspringgreen}{rgb}{0.09, 0.45, 0.27}
\definecolor{darkseagreen}{rgb}{0.56, 0.74, 0.56}
\definecolor{green(munsell)}{rgb}{0.0, 0.66, 0.47}
\pagecolor{white}
\color{black}
\maketitle
\section*{Exercise 3.5.15}
In each of the following, use the Euclidian algorithm to calculate the gcd and to find appropriate integers $x, y \in \mathbb{Z}$
\newline
(1) Find integers $x, y \in \mathbb{Z} \text{ such that } 547x+632y = \text{gcd}(547, 632)$
\newline
(2) Find integers $x, y \in \mathbb{Z} \text{ such that } 398x+620y = \text{gcd}(398, 600)$
\end{document}   

1. Exercise 3.5.15, (1) and (2).

2. Give an example of a function with two different left inverses. Give an example of another function that has a unique left inverse. Explain why the first function can have different left inverses, but the second can't.

3. Exercise 3.5.12.
