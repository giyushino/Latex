\documentclass[hidelinks]{article}
\usepackage[a4paper, total={7in, 10in}, top=0.25in]{geometry}
\usepackage[dvipsnames]{xcolor}
\usepackage{amsmath}
\usepackage{tikz}
\usepackage{tkz-euclide}
\usepackage[unicode]{hyperref}
\usepackage[all]{hypcap}
\usepackage{fancyhdr}
\usepackage{amsfonts}
\usepackage[utf8]{inputenc}
\DeclareUnicodeCharacter{2212}{-}
\usepackage{amsmath}
\usepackage{array}
\mathchardef\mhyphen="2D % Define a "math hyphen"
\newcommand\rnumber{\mathop{r\mhyphen number}}

\usetikzlibrary{angles,calc, decorations.pathreplacing}

\definecolor{carminered}{rgb}{1.0, 0.0, 0.22}
\definecolor{capri}{rgb}{0.0, 0.75, 1.0}
\definecolor{brightlavender}{rgb}{0.75, 0.58, 0.89}
\title{\textbf{Learning Roadmap}}
\author{Allan Zhang}
\date{April 2, 2025}
\begin{document}
\hypersetup{bookmarksnumbered=true,}
\definecolor{darkspringgreen}{rgb}{0.09, 0.45, 0.27}
\definecolor{darkseagreen}{rgb}{0.56, 0.74, 0.56}
\definecolor{green(munsell)}{rgb}{0.0, 0.66, 0.47}
\pagecolor{white}
\color{black}
\maketitle 
\section*{Major Focuses}
\begin{itemize}
    \item[a)] Building \textbf{intuition} is a crucial aspect of becoming proficient at mathematics; Rather than brute forcing problems, being able to recognize certain patterns in questions can simplify them tremendously. This principle isn't unique to math, but all subjects in general, even English.  
    \item[] To develop this intuition, we will do the following: 
    \begin{itemize}
        \item[1.)] Go through AMC 8 problems, specifically focusing on developing a solid mathematical foundation and becoming familiar with the different types of problems (i.e. basic combinatorics, number theory, geometry, algebra). If possible, I will teach her some of the more simple SAT math problems
        \item[2.)] Reinforce basic algebraic knowledge. Multiplication, division, exponent laws, logarithmic rules, fractions, etc. Prove each law ourselves
        \item[] $\quad$ Example of Zero Exponent Rule ``proof'': 
            \[
                a^0 = 1 \quad \quad a^0 = a^{n - n} \quad \quad a^{n - n} = \frac{a^n}{a^n} \quad \quad \frac{a^n}{a^n} = 1  
               \quad \quad \boxed{a^0 = 1}
            \]
        \item[] $\quad$ Example of developing mental math skills by solving 24 problems: 
            \[
            6, 7, 4, 1 \rightarrow 6 \cdot (7 + 1 - 4) = 6 \cdot 4 = 24
            \]
            \[
                1, 5, 5, 5 \rightarrow 5 \cdot ( 5 - \frac{1}{5}) =  
                5 \cdot \frac{25 - 1}{5} = 24 
            \]
            \[
                8, 8, 3, 3 \rightarrow 8 \div (3 - \frac{8}{3}) = 8 \div (\frac{1}{3}) = 8 \cdot 3 = 24  
            \]
    \end{itemize}
    
    \item[b)] \textbf{Deriving} different equations, formulas, and constants ourselves. Rather than forcing students to memorize abstract concepts that hold no real meaning to them, using the intuition we will developing, we'll learn how everything we are taught was derived in the past (if feasible/helpful)
    \begin{itemize}
        \item[Ex)] Derive the outputs of $\sin$ and $\cos$. Instead of memorizing the unit circle, we'll understand what it is and what the functions actually do to their input 
        \item[Ex)] Understanding where formulas like $a^2 + b^2 = c^2 $ and $\frac{n(n+1)}{2}$, the Pythagorean theorem and arithmetic series of $n$ integers, come from
    \end{itemize}

    \item[c)] Becoming familiar with \textbf{proof} based mathematics
    \begin{itemize}
        \item[a)] This won't be touched for a while, but some basic proofs and definitions can be taught at an early age
        \begin{itemize}
            \item[Ex)] The loose definition of a function could be stated as: a rule that takes in an input, and provides exactly one output. More formally, I would eventually like to introduce formal math vocabulary/notation. 
                \vspace{0.2cm}

For each \( x \in X \), there exists \( y \in Y \) such that \( (x, y) \in f \):
\[
\forall x \in X \, \exists y \in Y \, ((x, y) \in f)
\]

For each \( x \in X \), and for every \( y_1, y_2 \in Y \), if \( (x, y_1) \in f \) and \( (x, y_2) \in f \), then \( y_1 = y_2 \):

\[
\forall x \in X \, \forall y_1 \in Y \, \forall y_2 \in Y \, \left( (x, y_1) \in f \land (x, y_2) \in f \to y_1 = y_2 \right)
\]
        \item[ ] In other words, a function must have an output for every input and pass the vertical line test
        \end{itemize}
    \end{itemize}
\end{itemize}
\newpage
\section*{Current Notes/Focus}

From what I have observed from just one lesson, Anvitha seems to be a very intelligent child, no flattery intended. She picks up new concepts quickly and is able to apply what she has learned relatively easily. I think the one thing she lacks currently is an in-depth understanding of what exactly mathematical operators do, which isn't surprising due to how they are taught in school (or at least how I remember them to be taught). 
\vspace{0.2cm}

\noindent
For example, when I asked her what $10 \cdot 10 \cdot 10$ is, she answered with 100, even though she understands that multiplying a number by 10 adds a 0 to the end. Additionally, when I asked her to explain how she knew $\frac{2}{4}$ simplifies to $\frac{1}{2}$, she said she memorized that one. While there's no problem in memorizing things like that, I think it demonstrates a ``superficial'' understanding of what a fraction really is. 
\vspace{0.2cm}

\noindent
For the upcoming weeks, these are the main things I want to focus on: 
\begin{itemize}
    \item[1.)] Becoming proficient with fraction arithmetic (addition, subtraction, multiplication, division, multiplying by 1 in different forms, simplifying)
    \item[2.)] Understanding what decimals are and how decimal arithmetic differs from normal arithmetic. Additionally, go over relationship between division, fractions, and decimals
    \item[3.)] Learn associative and communative property 
    \item[4.)] Introduce variables and very basic prealgebra concepts 
\end{itemize}
\end{document}


